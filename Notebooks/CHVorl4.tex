
% Default to the notebook output style

    


% Inherit from the specified cell style.




    
\documentclass[10pt]{article}

    
    
    \usepackage[T1]{fontenc}
    % Nicer default font than Computer Modern for most use cases
    \usepackage{palatino}

    % Basic figure setup, for now with no caption control since it's done
    % automatically by Pandoc (which extracts ![](path) syntax from Markdown).
    \usepackage{graphicx}
    % We will generate all images so they have a width \maxwidth. This means
    % that they will get their normal width if they fit onto the page, but
    % are scaled down if they would overflow the margins.
    \makeatletter
    \def\maxwidth{\ifdim\Gin@nat@width>\linewidth\linewidth
    \else\Gin@nat@width\fi}
    \makeatother
    \let\Oldincludegraphics\includegraphics
    % Set max figure width to be 80% of text width, for now hardcoded.
    \renewcommand{\includegraphics}[1]{\Oldincludegraphics[width=.8\maxwidth]{#1}}
    % Ensure that by default, figures have no caption (until we provide a
    % proper Figure object with a Caption API and a way to capture that
    % in the conversion process - todo).
    \usepackage{caption}
    \DeclareCaptionLabelFormat{nolabel}{}
    \captionsetup{labelformat=nolabel}

    \usepackage{adjustbox} % Used to constrain images to a maximum size 
    \usepackage{xcolor} % Allow colors to be defined
    \usepackage{enumerate} % Needed for markdown enumerations to work
    \usepackage{geometry} % Used to adjust the document margins
    \usepackage{amsmath} % Equations
    \usepackage{amssymb} % Equations
    \usepackage{textcomp} % defines textquotesingle
    % Hack from http://tex.stackexchange.com/a/47451/13684:
    \AtBeginDocument{%
        \def\PYZsq{\textquotesingle}% Upright quotes in Pygmentized code
    }
    \usepackage{upquote} % Upright quotes for verbatim code
    \usepackage{eurosym} % defines \euro
    \usepackage[mathletters]{ucs} % Extended unicode (utf-8) support
    \usepackage[utf8x]{inputenc} % Allow utf-8 characters in the tex document
    \usepackage{fancyvrb} % verbatim replacement that allows latex
    \usepackage{grffile} % extends the file name processing of package graphics 
                         % to support a larger range 
    % The hyperref package gives us a pdf with properly built
    % internal navigation ('pdf bookmarks' for the table of contents,
    % internal cross-reference links, web links for URLs, etc.)
    \usepackage{hyperref}
    \usepackage{longtable} % longtable support required by pandoc >1.10
    \usepackage{booktabs}  % table support for pandoc > 1.12.2
    \usepackage[normalem]{ulem} % ulem is needed to support strikethroughs (\sout)
                                % normalem makes italics be italics, not underlines
    

    
    
    % Colors for the hyperref package
    \definecolor{urlcolor}{rgb}{0,.145,.698}
    \definecolor{linkcolor}{rgb}{.71,0.21,0.01}
    \definecolor{citecolor}{rgb}{.12,.54,.11}

    % ANSI colors
    \definecolor{ansi-black}{HTML}{3E424D}
    \definecolor{ansi-black-intense}{HTML}{282C36}
    \definecolor{ansi-red}{HTML}{E75C58}
    \definecolor{ansi-red-intense}{HTML}{B22B31}
    \definecolor{ansi-green}{HTML}{00A250}
    \definecolor{ansi-green-intense}{HTML}{007427}
    \definecolor{ansi-yellow}{HTML}{DDB62B}
    \definecolor{ansi-yellow-intense}{HTML}{B27D12}
    \definecolor{ansi-blue}{HTML}{208FFB}
    \definecolor{ansi-blue-intense}{HTML}{0065CA}
    \definecolor{ansi-magenta}{HTML}{D160C4}
    \definecolor{ansi-magenta-intense}{HTML}{A03196}
    \definecolor{ansi-cyan}{HTML}{60C6C8}
    \definecolor{ansi-cyan-intense}{HTML}{258F8F}
    \definecolor{ansi-white}{HTML}{C5C1B4}
    \definecolor{ansi-white-intense}{HTML}{A1A6B2}

    % commands and environments needed by pandoc snippets
    % extracted from the output of `pandoc -s`
    \providecommand{\tightlist}{%
      \setlength{\itemsep}{0pt}\setlength{\parskip}{0pt}}
    \DefineVerbatimEnvironment{Highlighting}{Verbatim}{commandchars=\\\{\}}
    % Add ',fontsize=\small' for more characters per line
    \newenvironment{Shaded}{}{}
    \newcommand{\KeywordTok}[1]{\textcolor[rgb]{0.00,0.44,0.13}{\textbf{{#1}}}}
    \newcommand{\DataTypeTok}[1]{\textcolor[rgb]{0.56,0.13,0.00}{{#1}}}
    \newcommand{\DecValTok}[1]{\textcolor[rgb]{0.25,0.63,0.44}{{#1}}}
    \newcommand{\BaseNTok}[1]{\textcolor[rgb]{0.25,0.63,0.44}{{#1}}}
    \newcommand{\FloatTok}[1]{\textcolor[rgb]{0.25,0.63,0.44}{{#1}}}
    \newcommand{\CharTok}[1]{\textcolor[rgb]{0.25,0.44,0.63}{{#1}}}
    \newcommand{\StringTok}[1]{\textcolor[rgb]{0.25,0.44,0.63}{{#1}}}
    \newcommand{\CommentTok}[1]{\textcolor[rgb]{0.38,0.63,0.69}{\textit{{#1}}}}
    \newcommand{\OtherTok}[1]{\textcolor[rgb]{0.00,0.44,0.13}{{#1}}}
    \newcommand{\AlertTok}[1]{\textcolor[rgb]{1.00,0.00,0.00}{\textbf{{#1}}}}
    \newcommand{\FunctionTok}[1]{\textcolor[rgb]{0.02,0.16,0.49}{{#1}}}
    \newcommand{\RegionMarkerTok}[1]{{#1}}
    \newcommand{\ErrorTok}[1]{\textcolor[rgb]{1.00,0.00,0.00}{\textbf{{#1}}}}
    \newcommand{\NormalTok}[1]{{#1}}
    
    % Additional commands for more recent versions of Pandoc
    \newcommand{\ConstantTok}[1]{\textcolor[rgb]{0.53,0.00,0.00}{{#1}}}
    \newcommand{\SpecialCharTok}[1]{\textcolor[rgb]{0.25,0.44,0.63}{{#1}}}
    \newcommand{\VerbatimStringTok}[1]{\textcolor[rgb]{0.25,0.44,0.63}{{#1}}}
    \newcommand{\SpecialStringTok}[1]{\textcolor[rgb]{0.73,0.40,0.53}{{#1}}}
    \newcommand{\ImportTok}[1]{{#1}}
    \newcommand{\DocumentationTok}[1]{\textcolor[rgb]{0.73,0.13,0.13}{\textit{{#1}}}}
    \newcommand{\AnnotationTok}[1]{\textcolor[rgb]{0.38,0.63,0.69}{\textbf{\textit{{#1}}}}}
    \newcommand{\CommentVarTok}[1]{\textcolor[rgb]{0.38,0.63,0.69}{\textbf{\textit{{#1}}}}}
    \newcommand{\VariableTok}[1]{\textcolor[rgb]{0.10,0.09,0.49}{{#1}}}
    \newcommand{\ControlFlowTok}[1]{\textcolor[rgb]{0.00,0.44,0.13}{\textbf{{#1}}}}
    \newcommand{\OperatorTok}[1]{\textcolor[rgb]{0.40,0.40,0.40}{{#1}}}
    \newcommand{\BuiltInTok}[1]{{#1}}
    \newcommand{\ExtensionTok}[1]{{#1}}
    \newcommand{\PreprocessorTok}[1]{\textcolor[rgb]{0.74,0.48,0.00}{{#1}}}
    \newcommand{\AttributeTok}[1]{\textcolor[rgb]{0.49,0.56,0.16}{{#1}}}
    \newcommand{\InformationTok}[1]{\textcolor[rgb]{0.38,0.63,0.69}{\textbf{\textit{{#1}}}}}
    \newcommand{\WarningTok}[1]{\textcolor[rgb]{0.38,0.63,0.69}{\textbf{\textit{{#1}}}}}
    
    
    % Define a nice break command that doesn't care if a line doesn't already
    % exist.
    \def\br{\hspace*{\fill} \\* }
    % Math Jax compatability definitions
    \def\gt{>}
    \def\lt{<}
    % Document parameters
    \title{CHVorl4}
    
    
    

    % Pygments definitions
    
\makeatletter
\def\PY@reset{\let\PY@it=\relax \let\PY@bf=\relax%
    \let\PY@ul=\relax \let\PY@tc=\relax%
    \let\PY@bc=\relax \let\PY@ff=\relax}
\def\PY@tok#1{\csname PY@tok@#1\endcsname}
\def\PY@toks#1+{\ifx\relax#1\empty\else%
    \PY@tok{#1}\expandafter\PY@toks\fi}
\def\PY@do#1{\PY@bc{\PY@tc{\PY@ul{%
    \PY@it{\PY@bf{\PY@ff{#1}}}}}}}
\def\PY#1#2{\PY@reset\PY@toks#1+\relax+\PY@do{#2}}

\expandafter\def\csname PY@tok@cs\endcsname{\let\PY@it=\textit\def\PY@tc##1{\textcolor[rgb]{0.25,0.50,0.50}{##1}}}
\expandafter\def\csname PY@tok@il\endcsname{\def\PY@tc##1{\textcolor[rgb]{0.40,0.40,0.40}{##1}}}
\expandafter\def\csname PY@tok@ch\endcsname{\let\PY@it=\textit\def\PY@tc##1{\textcolor[rgb]{0.25,0.50,0.50}{##1}}}
\expandafter\def\csname PY@tok@nt\endcsname{\let\PY@bf=\textbf\def\PY@tc##1{\textcolor[rgb]{0.00,0.50,0.00}{##1}}}
\expandafter\def\csname PY@tok@kn\endcsname{\let\PY@bf=\textbf\def\PY@tc##1{\textcolor[rgb]{0.00,0.50,0.00}{##1}}}
\expandafter\def\csname PY@tok@ss\endcsname{\def\PY@tc##1{\textcolor[rgb]{0.10,0.09,0.49}{##1}}}
\expandafter\def\csname PY@tok@gh\endcsname{\let\PY@bf=\textbf\def\PY@tc##1{\textcolor[rgb]{0.00,0.00,0.50}{##1}}}
\expandafter\def\csname PY@tok@mo\endcsname{\def\PY@tc##1{\textcolor[rgb]{0.40,0.40,0.40}{##1}}}
\expandafter\def\csname PY@tok@nf\endcsname{\def\PY@tc##1{\textcolor[rgb]{0.00,0.00,1.00}{##1}}}
\expandafter\def\csname PY@tok@k\endcsname{\let\PY@bf=\textbf\def\PY@tc##1{\textcolor[rgb]{0.00,0.50,0.00}{##1}}}
\expandafter\def\csname PY@tok@bp\endcsname{\def\PY@tc##1{\textcolor[rgb]{0.00,0.50,0.00}{##1}}}
\expandafter\def\csname PY@tok@nn\endcsname{\let\PY@bf=\textbf\def\PY@tc##1{\textcolor[rgb]{0.00,0.00,1.00}{##1}}}
\expandafter\def\csname PY@tok@gr\endcsname{\def\PY@tc##1{\textcolor[rgb]{1.00,0.00,0.00}{##1}}}
\expandafter\def\csname PY@tok@gs\endcsname{\let\PY@bf=\textbf}
\expandafter\def\csname PY@tok@m\endcsname{\def\PY@tc##1{\textcolor[rgb]{0.40,0.40,0.40}{##1}}}
\expandafter\def\csname PY@tok@si\endcsname{\let\PY@bf=\textbf\def\PY@tc##1{\textcolor[rgb]{0.73,0.40,0.53}{##1}}}
\expandafter\def\csname PY@tok@sb\endcsname{\def\PY@tc##1{\textcolor[rgb]{0.73,0.13,0.13}{##1}}}
\expandafter\def\csname PY@tok@sc\endcsname{\def\PY@tc##1{\textcolor[rgb]{0.73,0.13,0.13}{##1}}}
\expandafter\def\csname PY@tok@no\endcsname{\def\PY@tc##1{\textcolor[rgb]{0.53,0.00,0.00}{##1}}}
\expandafter\def\csname PY@tok@s1\endcsname{\def\PY@tc##1{\textcolor[rgb]{0.73,0.13,0.13}{##1}}}
\expandafter\def\csname PY@tok@gi\endcsname{\def\PY@tc##1{\textcolor[rgb]{0.00,0.63,0.00}{##1}}}
\expandafter\def\csname PY@tok@kr\endcsname{\let\PY@bf=\textbf\def\PY@tc##1{\textcolor[rgb]{0.00,0.50,0.00}{##1}}}
\expandafter\def\csname PY@tok@mb\endcsname{\def\PY@tc##1{\textcolor[rgb]{0.40,0.40,0.40}{##1}}}
\expandafter\def\csname PY@tok@mf\endcsname{\def\PY@tc##1{\textcolor[rgb]{0.40,0.40,0.40}{##1}}}
\expandafter\def\csname PY@tok@nd\endcsname{\def\PY@tc##1{\textcolor[rgb]{0.67,0.13,1.00}{##1}}}
\expandafter\def\csname PY@tok@nv\endcsname{\def\PY@tc##1{\textcolor[rgb]{0.10,0.09,0.49}{##1}}}
\expandafter\def\csname PY@tok@cpf\endcsname{\let\PY@it=\textit\def\PY@tc##1{\textcolor[rgb]{0.25,0.50,0.50}{##1}}}
\expandafter\def\csname PY@tok@se\endcsname{\let\PY@bf=\textbf\def\PY@tc##1{\textcolor[rgb]{0.73,0.40,0.13}{##1}}}
\expandafter\def\csname PY@tok@ge\endcsname{\let\PY@it=\textit}
\expandafter\def\csname PY@tok@err\endcsname{\def\PY@bc##1{\setlength{\fboxsep}{0pt}\fcolorbox[rgb]{1.00,0.00,0.00}{1,1,1}{\strut ##1}}}
\expandafter\def\csname PY@tok@gp\endcsname{\let\PY@bf=\textbf\def\PY@tc##1{\textcolor[rgb]{0.00,0.00,0.50}{##1}}}
\expandafter\def\csname PY@tok@s2\endcsname{\def\PY@tc##1{\textcolor[rgb]{0.73,0.13,0.13}{##1}}}
\expandafter\def\csname PY@tok@kp\endcsname{\def\PY@tc##1{\textcolor[rgb]{0.00,0.50,0.00}{##1}}}
\expandafter\def\csname PY@tok@na\endcsname{\def\PY@tc##1{\textcolor[rgb]{0.49,0.56,0.16}{##1}}}
\expandafter\def\csname PY@tok@cp\endcsname{\def\PY@tc##1{\textcolor[rgb]{0.74,0.48,0.00}{##1}}}
\expandafter\def\csname PY@tok@go\endcsname{\def\PY@tc##1{\textcolor[rgb]{0.53,0.53,0.53}{##1}}}
\expandafter\def\csname PY@tok@gu\endcsname{\let\PY@bf=\textbf\def\PY@tc##1{\textcolor[rgb]{0.50,0.00,0.50}{##1}}}
\expandafter\def\csname PY@tok@c\endcsname{\let\PY@it=\textit\def\PY@tc##1{\textcolor[rgb]{0.25,0.50,0.50}{##1}}}
\expandafter\def\csname PY@tok@gd\endcsname{\def\PY@tc##1{\textcolor[rgb]{0.63,0.00,0.00}{##1}}}
\expandafter\def\csname PY@tok@vc\endcsname{\def\PY@tc##1{\textcolor[rgb]{0.10,0.09,0.49}{##1}}}
\expandafter\def\csname PY@tok@cm\endcsname{\let\PY@it=\textit\def\PY@tc##1{\textcolor[rgb]{0.25,0.50,0.50}{##1}}}
\expandafter\def\csname PY@tok@nc\endcsname{\let\PY@bf=\textbf\def\PY@tc##1{\textcolor[rgb]{0.00,0.00,1.00}{##1}}}
\expandafter\def\csname PY@tok@sx\endcsname{\def\PY@tc##1{\textcolor[rgb]{0.00,0.50,0.00}{##1}}}
\expandafter\def\csname PY@tok@gt\endcsname{\def\PY@tc##1{\textcolor[rgb]{0.00,0.27,0.87}{##1}}}
\expandafter\def\csname PY@tok@c1\endcsname{\let\PY@it=\textit\def\PY@tc##1{\textcolor[rgb]{0.25,0.50,0.50}{##1}}}
\expandafter\def\csname PY@tok@ow\endcsname{\let\PY@bf=\textbf\def\PY@tc##1{\textcolor[rgb]{0.67,0.13,1.00}{##1}}}
\expandafter\def\csname PY@tok@nb\endcsname{\def\PY@tc##1{\textcolor[rgb]{0.00,0.50,0.00}{##1}}}
\expandafter\def\csname PY@tok@vg\endcsname{\def\PY@tc##1{\textcolor[rgb]{0.10,0.09,0.49}{##1}}}
\expandafter\def\csname PY@tok@sh\endcsname{\def\PY@tc##1{\textcolor[rgb]{0.73,0.13,0.13}{##1}}}
\expandafter\def\csname PY@tok@s\endcsname{\def\PY@tc##1{\textcolor[rgb]{0.73,0.13,0.13}{##1}}}
\expandafter\def\csname PY@tok@ne\endcsname{\let\PY@bf=\textbf\def\PY@tc##1{\textcolor[rgb]{0.82,0.25,0.23}{##1}}}
\expandafter\def\csname PY@tok@mi\endcsname{\def\PY@tc##1{\textcolor[rgb]{0.40,0.40,0.40}{##1}}}
\expandafter\def\csname PY@tok@o\endcsname{\def\PY@tc##1{\textcolor[rgb]{0.40,0.40,0.40}{##1}}}
\expandafter\def\csname PY@tok@nl\endcsname{\def\PY@tc##1{\textcolor[rgb]{0.63,0.63,0.00}{##1}}}
\expandafter\def\csname PY@tok@sd\endcsname{\let\PY@it=\textit\def\PY@tc##1{\textcolor[rgb]{0.73,0.13,0.13}{##1}}}
\expandafter\def\csname PY@tok@kt\endcsname{\def\PY@tc##1{\textcolor[rgb]{0.69,0.00,0.25}{##1}}}
\expandafter\def\csname PY@tok@sr\endcsname{\def\PY@tc##1{\textcolor[rgb]{0.73,0.40,0.53}{##1}}}
\expandafter\def\csname PY@tok@kc\endcsname{\let\PY@bf=\textbf\def\PY@tc##1{\textcolor[rgb]{0.00,0.50,0.00}{##1}}}
\expandafter\def\csname PY@tok@vi\endcsname{\def\PY@tc##1{\textcolor[rgb]{0.10,0.09,0.49}{##1}}}
\expandafter\def\csname PY@tok@ni\endcsname{\let\PY@bf=\textbf\def\PY@tc##1{\textcolor[rgb]{0.60,0.60,0.60}{##1}}}
\expandafter\def\csname PY@tok@kd\endcsname{\let\PY@bf=\textbf\def\PY@tc##1{\textcolor[rgb]{0.00,0.50,0.00}{##1}}}
\expandafter\def\csname PY@tok@w\endcsname{\def\PY@tc##1{\textcolor[rgb]{0.73,0.73,0.73}{##1}}}
\expandafter\def\csname PY@tok@mh\endcsname{\def\PY@tc##1{\textcolor[rgb]{0.40,0.40,0.40}{##1}}}

\def\PYZbs{\char`\\}
\def\PYZus{\char`\_}
\def\PYZob{\char`\{}
\def\PYZcb{\char`\}}
\def\PYZca{\char`\^}
\def\PYZam{\char`\&}
\def\PYZlt{\char`\<}
\def\PYZgt{\char`\>}
\def\PYZsh{\char`\#}
\def\PYZpc{\char`\%}
\def\PYZdl{\char`\$}
\def\PYZhy{\char`\-}
\def\PYZsq{\char`\'}
\def\PYZdq{\char`\"}
\def\PYZti{\char`\~}
% for compatibility with earlier versions
\def\PYZat{@}
\def\PYZlb{[}
\def\PYZrb{]}
\makeatother


    % Exact colors from NB
    \definecolor{incolor}{rgb}{0.0, 0.0, 0.5}
    \definecolor{outcolor}{rgb}{0.545, 0.0, 0.0}



    
    % Prevent overflowing lines due to hard-to-break entities
    \sloppy 
    % Setup hyperref package
    \hypersetup{
      breaklinks=true,  % so long urls are correctly broken across lines
      colorlinks=true,
      urlcolor=urlcolor,
      linkcolor=linkcolor,
      citecolor=citecolor,
      }
    % Slightly bigger margins than the latex defaults
    
    \geometry{verbose,tmargin=1in,bmargin=1in,lmargin=1in,rmargin=1in}
    
    

    \begin{document}
    
    
    \maketitle
    
    

    
    \section{Vorlesung 4: Strukturen von Texten
2}\label{vorlesung-4-strukturen-von-texten-2}

Reguläre Ausdrücke: http://www.regexe.de/hilfe.jsp
https://pymotw.com/2/re

Pandas: http://www.data-analysis-in-python.org/3\_pandas.html :
https://bitbucket.org/hrojas/learn-pandas

    \begin{Verbatim}[commandchars=\\\{\}]
{\color{incolor}In [{\color{incolor}1}]:} \PY{k+kn}{import} \PY{n+nn}{json}
        \PY{k+kn}{import} \PY{n+nn}{pandas} \PY{k}{as} \PY{n+nn}{pd}
        \PY{k+kn}{import} \PY{n+nn}{re}
        \PY{k+kn}{import} \PY{n+nn}{numpy} \PY{k}{as} \PY{n+nn}{np}
\end{Verbatim}

    \subsection{\texorpdfstring{Text als Dataframe
\emph{Poleis}}{Text als Dataframe Poleis}}\label{text-als-dataframe-poleis}

    \begin{Verbatim}[commandchars=\\\{\}]
{\color{incolor}In [{\color{incolor}2}]:} \PY{c+c1}{\PYZsh{} json.load erzeugt ein dictionary der JSON Daten}
        \PY{k}{with} \PY{n+nb}{open}\PY{p}{(}\PY{l+s+s1}{\PYZsq{}}\PY{l+s+s1}{chapter1.json}\PY{l+s+s1}{\PYZsq{}}\PY{p}{)} \PY{k}{as} \PY{n}{json\PYZus{}data}\PY{p}{:}
            \PY{n}{PoleisRawData} \PY{o}{=} \PY{n}{json}\PY{o}{.}\PY{n}{load}\PY{p}{(}\PY{n}{json\PYZus{}data}\PY{p}{)}
\end{Verbatim}

    \begin{Verbatim}[commandchars=\\\{\}]
{\color{incolor}In [{\color{incolor}3}]:} \PY{c+c1}{\PYZsh{} list() erstellt eine Liste der keys}
        \PY{n}{PoleisKeyList} \PY{o}{=} \PY{n+nb}{list}\PY{p}{(}\PY{n}{PoleisRawData}\PY{o}{.}\PY{n}{keys}\PY{p}{(}\PY{p}{)}\PY{p}{)}
        \PY{n}{PoleisKeyList}
\end{Verbatim}

            \begin{Verbatim}[commandchars=\\\{\}]
{\color{outcolor}Out[{\color{outcolor}3}]:} ['18. Heloron ',
         '41. Naxos ',
         '11. Alaisa ',
         '34. Lipara ',
         '23. Herbita ',
         '48. Tauromenion ',
         '25. Hippana ',
         '21. Herakleia 2 ',
         '29. Kasmenai ',
         '30. Katane ',
         '27. Kallipolis ',
         '42. Petra ',
         '37. Morgantina ',
         '9. Akragas ',
         '44. Selinous ',
         '24. Himera ',
         '13. Apollonia ',
         '17. Gela ',
         '8. Aitna ',
         '36. Megara ',
         '35. *Longane ',
         '33. Leontinoi ',
         '38. Mylai ',
         '49. Tyndaris ',
         '39. Mytistratos ',
         '12. Alontion ',
         '32. Kephaloidion ',
         '40. Nakone ',
         '26 *Imachara ',
         '19. Henna ',
         '14. Engyon ',
         '16. Galeria ',
         '47. Syrakousai ',
         '5. Abakainon ',
         '50. (Tyrrhenoi)',
         '22. Herbes(s)os ',
         '20. Herakleia 1',
         '6. Adranon ',
         '15. Euboia ',
         '31. Kentoripa ',
         '43. Piakos ',
         '7. Agyrion ',
         '10. Akrai ',
         '45. (Sileraioi)',
         '28. Kamarina ',
         '51. Zankle ',
         '46. (Stielanaioi)']
\end{Verbatim}
        
    \begin{Verbatim}[commandchars=\\\{\}]
{\color{incolor}In [{\color{incolor}4}]:} \PY{c+c1}{\PYZsh{}\PYZpc{}\PYZpc{}tutor \PYZhy{}\PYZhy{}lang python3}
        \PY{k}{for} \PY{n}{i} \PY{o+ow}{in} \PY{n}{PoleisKeyList}\PY{p}{:}
            \PY{k}{if} \PY{l+s+s2}{\PYZdq{}}\PY{l+s+s2}{Megara}\PY{l+s+s2}{\PYZdq{}} \PY{o+ow}{in} \PY{n}{PoleisRawData}\PY{p}{[}\PY{n}{i}\PY{p}{]}\PY{p}{:}
                \PY{n+nb}{print}\PY{p}{(}\PY{n}{i}\PY{p}{)}
\end{Verbatim}

    \begin{Verbatim}[commandchars=\\\{\}]
18. Heloron 
44. Selinous 
36. Megara 
33. Leontinoi 
47. Syrakousai 
15. Euboia 
46. (Stielanaioi)

    \end{Verbatim}

    \begin{Verbatim}[commandchars=\\\{\}]
{\color{incolor}In [{\color{incolor}5}]:} \PY{c+c1}{\PYZsh{} Liest das Dictionary als Dataframe ein. Namen der Poleis werden als Index benutzt.}
        \PY{n}{dfPoleis} \PY{o}{=} \PY{n}{pd}\PY{o}{.}\PY{n}{DataFrame}\PY{p}{(}\PY{p}{[}\PY{n}{PoleisRawData}\PY{p}{]}\PY{p}{)}\PY{o}{.}\PY{n}{transpose}\PY{p}{(}\PY{p}{)}
        \PY{n}{dfPoleis} \PY{o}{=} \PY{n}{dfPoleis}\PY{o}{.}\PY{n}{rename}\PY{p}{(}\PY{n}{columns}\PY{o}{=}\PY{p}{\PYZob{}}\PY{l+m+mi}{0}\PY{p}{:} \PY{l+s+s1}{\PYZsq{}}\PY{l+s+s1}{Beschreibung}\PY{l+s+s1}{\PYZsq{}}\PY{p}{\PYZcb{}}\PY{p}{)}
        \PY{n}{dfPoleis}\PY{o}{.}\PY{n}{head}\PY{p}{(}\PY{p}{)}
\end{Verbatim}

            \begin{Verbatim}[commandchars=\\\{\}]
{\color{outcolor}Out[{\color{outcolor}5}]:}                                                      Beschreibung
        10. Akrai       (Akraios) Map  47.  Lat. 37.05,long.  14.55.  {\ldots}
        11. Alaisa      (Alaisinos) Map  47.  Lat. 38.00,long.  14.15{\ldots}
        12. Alontion    (Alontinos) Map  47.  Lat. 38.05,long.  14.40{\ldots}
        13. Apollonia   (Apolloniates) Map  47.  Lat. 38.00,long. 14.3{\ldots}
        14. Engyon      (Engyinos)  Map  47.  Lat. 37.45,long.  14.35{\ldots}
\end{Verbatim}
        
    \section{Konstruktion neuer Merkmale}\label{konstruktion-neuer-merkmale}

\subsection{Textmuster mit regulären
Ausdrücken}\label{textmuster-mit-reguluxe4ren-ausdruxfccken}

http://www.regexe.de/hilfe.jsp
https://www.cheatography.com/davechild/cheat-sheets/regular-expressions/

http://www.coli.uni-saarland.de/courses/python1-10/folien/PythonI10-07.pdf

    \begin{Verbatim}[commandchars=\\\{\}]
{\color{incolor}In [{\color{incolor}6}]:} \PY{c+c1}{\PYZsh{} Konstruktion einer Liste mit sogenannten List\PYZhy{}Comprehensions}
        \PY{n}{ListCities} \PY{o}{=} \PY{p}{[}\PY{n}{x}\PY{p}{[}\PY{l+m+mi}{4}\PY{p}{:}\PY{p}{]} \PY{k}{for} \PY{n}{x} \PY{o+ow}{in} \PY{n}{dfPoleis}\PY{o}{.}\PY{n}{index}\PY{p}{]}
        \PY{n}{ListCities}
\end{Verbatim}

            \begin{Verbatim}[commandchars=\\\{\}]
{\color{outcolor}Out[{\color{outcolor}6}]:} ['Akrai ',
         'Alaisa ',
         'Alontion ',
         'Apollonia ',
         'Engyon ',
         'Euboia ',
         'Galeria ',
         'Gela ',
         'Heloron ',
         'Henna ',
         'Herakleia 1',
         'Herakleia 2 ',
         'Herbes(s)os ',
         'Herbita ',
         'Himera ',
         'Hippana ',
         'Imachara ',
         'Kallipolis ',
         'Kamarina ',
         'Kasmenai ',
         'Katane ',
         'Kentoripa ',
         'Kephaloidion ',
         'Leontinoi ',
         'Lipara ',
         '*Longane ',
         'Megara ',
         'Morgantina ',
         'Mylai ',
         'Mytistratos ',
         'Nakone ',
         'Naxos ',
         'Petra ',
         'Piakos ',
         'Selinous ',
         '(Sileraioi)',
         '(Stielanaioi)',
         'Syrakousai ',
         'Tauromenion ',
         'Tyndaris ',
         'bakainon ',
         '(Tyrrhenoi)',
         'Zankle ',
         'dranon ',
         'gyrion ',
         'itna ',
         'kragas ']
\end{Verbatim}
        
    \begin{Verbatim}[commandchars=\\\{\}]
{\color{incolor}In [{\color{incolor}7}]:} \PY{c+c1}{\PYZsh{} Extrahiere Name der Polis aus Index}
        \PY{n}{dfPoleis}\PY{p}{[}\PY{l+s+s1}{\PYZsq{}}\PY{l+s+s1}{city}\PY{l+s+s1}{\PYZsq{}}\PY{p}{]} \PY{o}{=} \PY{p}{[}\PY{n}{x}\PY{p}{[}\PY{l+m+mi}{4}\PY{p}{:}\PY{p}{]} \PY{k}{for} \PY{n}{x} \PY{o+ow}{in} \PY{n}{dfPoleis}\PY{o}{.}\PY{n}{index}\PY{p}{]}
        \PY{n}{dfPoleis}\PY{o}{.}\PY{n}{head}\PY{p}{(}\PY{p}{)}
\end{Verbatim}

            \begin{Verbatim}[commandchars=\\\{\}]
{\color{outcolor}Out[{\color{outcolor}7}]:}                                                      Beschreibung        city
        10. Akrai       (Akraios) Map  47.  Lat. 37.05,long.  14.55.  {\ldots}      Akrai 
        11. Alaisa      (Alaisinos) Map  47.  Lat. 38.00,long.  14.15{\ldots}     Alaisa 
        12. Alontion    (Alontinos) Map  47.  Lat. 38.05,long.  14.40{\ldots}   Alontion 
        13. Apollonia   (Apolloniates) Map  47.  Lat. 38.00,long. 14.3{\ldots}  Apollonia 
        14. Engyon      (Engyinos)  Map  47.  Lat. 37.45,long.  14.35{\ldots}     Engyon 
\end{Verbatim}
        
    \begin{Verbatim}[commandchars=\\\{\}]
{\color{incolor}In [{\color{incolor}9}]:} \PY{p}{[}\PY{n}{re}\PY{o}{.}\PY{n}{findall}\PY{p}{(}\PY{l+s+s2}{\PYZdq{}}\PY{l+s+s2}{\PYZbs{}}\PY{l+s+s2}{d}\PY{l+s+s2}{\PYZob{}}\PY{l+s+s2}{1,2\PYZcb{}}\PY{l+s+s2}{\PYZdq{}}\PY{p}{,}\PY{n}{x}\PY{p}{)} \PY{k}{for} \PY{n}{x} \PY{o+ow}{in} \PY{n}{dfPoleis}\PY{o}{.}\PY{n}{index}\PY{p}{]}
\end{Verbatim}

            \begin{Verbatim}[commandchars=\\\{\}]
{\color{outcolor}Out[{\color{outcolor}9}]:} [['10'],
         ['11'],
         ['12'],
         ['13'],
         ['14'],
         ['15'],
         ['16'],
         ['17'],
         ['18'],
         ['19'],
         ['20', '1'],
         ['21', '2'],
         ['22'],
         ['23'],
         ['24'],
         ['25'],
         ['26'],
         ['27'],
         ['28'],
         ['29'],
         ['30'],
         ['31'],
         ['32'],
         ['33'],
         ['34'],
         ['35'],
         ['36'],
         ['37'],
         ['38'],
         ['39'],
         ['40'],
         ['41'],
         ['42'],
         ['43'],
         ['44'],
         ['45'],
         ['46'],
         ['47'],
         ['48'],
         ['49'],
         ['5'],
         ['50'],
         ['51'],
         ['6'],
         ['7'],
         ['8'],
         ['9']]
\end{Verbatim}
        
    \begin{Verbatim}[commandchars=\\\{\}]
{\color{incolor}In [{\color{incolor}10}]:} \PY{c+c1}{\PYZsh{} Extrahiere Nummer des Polis Eintrags}
         \PY{n}{dfPoleis}\PY{p}{[}\PY{l+s+s1}{\PYZsq{}}\PY{l+s+s1}{city\PYZus{}index}\PY{l+s+s1}{\PYZsq{}}\PY{p}{]} \PY{o}{=} \PY{p}{[}\PY{n+nb}{int}\PY{p}{(}\PY{n}{re}\PY{o}{.}\PY{n}{findall}\PY{p}{(}\PY{l+s+s1}{\PYZsq{}}\PY{l+s+s1}{\PYZbs{}}\PY{l+s+s1}{d}\PY{l+s+s1}{\PYZob{}}\PY{l+s+s1}{1,2\PYZcb{}}\PY{l+s+s1}{\PYZsq{}}\PY{p}{,} \PY{n}{x}\PY{p}{)}\PY{p}{[}\PY{l+m+mi}{0}\PY{p}{]}\PY{p}{)} \PY{k}{for} \PY{n}{x} \PY{o+ow}{in} \PY{n}{dfPoleis}\PY{o}{.}\PY{n}{index}\PY{p}{]}
         \PY{n}{dfPoleis}\PY{o}{.}\PY{n}{head}\PY{p}{(}\PY{p}{)}
\end{Verbatim}

            \begin{Verbatim}[commandchars=\\\{\}]
{\color{outcolor}Out[{\color{outcolor}10}]:}                                                      Beschreibung        city  \textbackslash{}
         10. Akrai       (Akraios) Map  47.  Lat. 37.05,long.  14.55.  {\ldots}      Akrai    
         11. Alaisa      (Alaisinos) Map  47.  Lat. 38.00,long.  14.15{\ldots}     Alaisa    
         12. Alontion    (Alontinos) Map  47.  Lat. 38.05,long.  14.40{\ldots}   Alontion    
         13. Apollonia   (Apolloniates) Map  47.  Lat. 38.00,long. 14.3{\ldots}  Apollonia    
         14. Engyon      (Engyinos)  Map  47.  Lat. 37.45,long.  14.35{\ldots}     Engyon    
         
                         city\_index  
         10. Akrai               10  
         11. Alaisa              11  
         12. Alontion            12  
         13. Apollonia           13  
         14. Engyon              14  
\end{Verbatim}
        
    \begin{Verbatim}[commandchars=\\\{\}]
{\color{incolor}In [{\color{incolor}11}]:} \PY{c+c1}{\PYZsh{} Sortiere die Zeilen nach der Spalte }
         \PY{n}{dfPoleis} \PY{o}{=} \PY{n}{dfPoleis}\PY{o}{.}\PY{n}{sort\PYZus{}values}\PY{p}{(}\PY{n}{by}\PY{o}{=}\PY{l+s+s1}{\PYZsq{}}\PY{l+s+s1}{city\PYZus{}index}\PY{l+s+s1}{\PYZsq{}}\PY{p}{)}
         \PY{n}{dfPoleis}\PY{o}{.}\PY{n}{head}\PY{p}{(}\PY{l+m+mi}{4}\PY{p}{)}
\end{Verbatim}

            \begin{Verbatim}[commandchars=\\\{\}]
{\color{outcolor}Out[{\color{outcolor}11}]:}                                                     Beschreibung       city  \textbackslash{}
         5. Abakainon   (Abakaininos) Map  47.  Lat. 38.05,long. 15.05{\ldots}  bakainon    
         6. Adranon     (Adranites) Map  47.  Lat. 37.40,long.  14.50{\ldots}    dranon    
         7. Agyrion     (Agyrinaios) Map  47.  Lat. 37.40,long.  14.30{\ldots}    gyrion    
         8. Aitna       (Aitnaios) Map  47.Location  of  Aitna  I  as {\ldots}      itna    
         
                        city\_index  
         5. Abakainon            5  
         6. Adranon              6  
         7. Agyrion              7  
         8. Aitna                8  
\end{Verbatim}
        
    \subsection{Textmustersuche in der Beschreibung einer
Polis}\label{textmustersuche-in-der-beschreibung-einer-polis}

    \subsubsection{Neue Funktionen}\label{neue-funktionen}

    \begin{Verbatim}[commandchars=\\\{\}]
{\color{incolor}In [{\color{incolor}12}]:} \PY{n}{nl}\PY{o}{=}\PY{n}{dfPoleis}\PY{p}{[}\PY{l+s+s2}{\PYZdq{}}\PY{l+s+s2}{city}\PY{l+s+s2}{\PYZdq{}}\PY{p}{]}
         \PY{n}{nl}\PY{o}{.}\PY{n}{head}\PY{p}{(}\PY{p}{)}
\end{Verbatim}

            \begin{Verbatim}[commandchars=\\\{\}]
{\color{outcolor}Out[{\color{outcolor}12}]:} 5. Abakainon     bakainon 
         6. Adranon         dranon 
         7. Agyrion         gyrion 
         8. Aitna             itna 
         9. Akragas         kragas 
         Name: city, dtype: object
\end{Verbatim}
        
    \begin{Verbatim}[commandchars=\\\{\}]
{\color{incolor}In [{\color{incolor}13}]:} \PY{n}{lcity}\PY{o}{=}\PY{n+nb}{list}\PY{p}{(}\PY{n}{dfPoleis}\PY{p}{[}\PY{l+s+s2}{\PYZdq{}}\PY{l+s+s2}{city}\PY{l+s+s2}{\PYZdq{}}\PY{p}{]}\PY{p}{)}
         \PY{n}{lcity}\PY{p}{[}\PY{l+m+mi}{0}\PY{p}{:}\PY{l+m+mi}{3}\PY{p}{]}
\end{Verbatim}

            \begin{Verbatim}[commandchars=\\\{\}]
{\color{outcolor}Out[{\color{outcolor}13}]:} ['bakainon ', 'dranon ', 'gyrion ']
\end{Verbatim}
        
    \begin{Verbatim}[commandchars=\\\{\}]
{\color{incolor}In [{\color{incolor}14}]:} \PY{k}{for} \PY{n}{i} \PY{o+ow}{in} \PY{n}{lcity}\PY{p}{:}
             \PY{k}{if} \PY{n}{re}\PY{o}{.}\PY{n}{search}\PY{p}{(}\PY{l+s+s1}{\PYZsq{}}\PY{l+s+s1}{oi}\PY{l+s+s1}{\PYZsq{}}\PY{p}{,}\PY{n}{i}\PY{p}{)}\PY{p}{:}
                 \PY{n+nb}{print}\PY{p}{(}\PY{n}{i}\PY{p}{)}
\end{Verbatim}

    \begin{Verbatim}[commandchars=\\\{\}]
Euboia 
Kephaloidion 
Leontinoi 
(Sileraioi)
(Stielanaioi)
(Tyrrhenoi)

    \end{Verbatim}

    \begin{Verbatim}[commandchars=\\\{\}]
{\color{incolor}In [{\color{incolor}15}]:} \PY{n}{dfPoleis}\PY{p}{[}\PY{n}{dfPoleis}\PY{p}{[}\PY{l+s+s2}{\PYZdq{}}\PY{l+s+s2}{city}\PY{l+s+s2}{\PYZdq{}}\PY{p}{]}\PY{o}{.}\PY{n}{str}\PY{o}{.}\PY{n}{contains}\PY{p}{(}\PY{l+s+s2}{\PYZdq{}}\PY{l+s+s2}{ag}\PY{l+s+s2}{\PYZdq{}}\PY{p}{)}\PY{p}{]}
\end{Verbatim}

            \begin{Verbatim}[commandchars=\\\{\}]
{\color{outcolor}Out[{\color{outcolor}15}]:}                                                   Beschreibung     city  \textbackslash{}
         9. Akragas   (Akragantinos) Map  47.  Lat. 37.20,long.  13{\ldots}  kragas    
         
                      city\_index  
         9. Akragas            9  
\end{Verbatim}
        
    \begin{Verbatim}[commandchars=\\\{\}]
{\color{incolor}In [{\color{incolor}16}]:} \PY{n}{dfNeu}\PY{o}{=}\PY{n}{dfPoleis}\PY{p}{[}\PY{n}{dfPoleis}\PY{p}{[}\PY{l+s+s2}{\PYZdq{}}\PY{l+s+s2}{city}\PY{l+s+s2}{\PYZdq{}}\PY{p}{]}\PY{o}{.}\PY{n}{str}\PY{o}{.}\PY{n}{contains}\PY{p}{(}\PY{l+s+s2}{\PYZdq{}}\PY{l+s+s2}{ag}\PY{l+s+s2}{\PYZdq{}}\PY{p}{)}\PY{p}{]}
\end{Verbatim}

    \subsubsection{Geographische
Koordinaten}\label{geographische-koordinaten}

    \begin{Verbatim}[commandchars=\\\{\}]
{\color{incolor}In [{\color{incolor}17}]:} \PY{k}{def} \PY{n+nf}{ListePattern}\PY{p}{(}\PY{n}{string}\PY{p}{,}\PY{n}{pattern}\PY{p}{)}\PY{p}{:}
             \PY{n}{x} \PY{o}{=} \PY{n}{re}\PY{o}{.}\PY{n}{findall}\PY{p}{(}\PY{n}{pattern}\PY{p}{,}\PY{n}{string}\PY{p}{)}
             \PY{k}{if} \PY{n}{x}\PY{p}{:}
                 \PY{k}{return}\PY{p}{(}\PY{n}{x}\PY{p}{)}
\end{Verbatim}

    \begin{itemize}
\tightlist
\item
  (?\textless{}=Lat.\s) Group (?\ldots{}) Passive (non-capturing) group
\item
  ?\textless{}= Lookbehind assertion
\item
  Lat.\s   das string muster: ``Lat.'' mit ``.'' und " " als escape
\item
  \s?\d+.\d+ : space{[}optional wegen ?{]}digit{[}1 oder 2 wegen
  +{]}.{[}escaped{]}digit{[}1 oder 2{]}
\end{itemize}

    \begin{Verbatim}[commandchars=\\\{\}]
{\color{incolor}In [{\color{incolor}18}]:} \PY{n}{ListePattern}\PY{p}{(}\PY{n}{dfPoleis}\PY{p}{[}\PY{l+s+s2}{\PYZdq{}}\PY{l+s+s2}{Beschreibung}\PY{l+s+s2}{\PYZdq{}}\PY{p}{]}\PY{p}{[}\PY{l+m+mi}{0}\PY{p}{]}\PY{p}{,}\PY{l+s+s2}{\PYZdq{}}\PY{l+s+s2}{(?\PYZlt{}=Lat}\PY{l+s+s2}{\PYZbs{}}\PY{l+s+s2}{.}\PY{l+s+s2}{\PYZbs{}}\PY{l+s+s2}{s)}\PY{l+s+s2}{\PYZbs{}}\PY{l+s+s2}{s?}\PY{l+s+s2}{\PYZbs{}}\PY{l+s+s2}{d+}\PY{l+s+s2}{\PYZbs{}}\PY{l+s+s2}{.}\PY{l+s+s2}{\PYZbs{}}\PY{l+s+s2}{d+}\PY{l+s+s2}{\PYZdq{}}\PY{p}{)}
\end{Verbatim}

            \begin{Verbatim}[commandchars=\\\{\}]
{\color{outcolor}Out[{\color{outcolor}18}]:} ['38.05']
\end{Verbatim}
        
    \begin{Verbatim}[commandchars=\\\{\}]
{\color{incolor}In [{\color{incolor}19}]:} \PY{c+c1}{\PYZsh{} gleiches auch für long.}
         \PY{n}{listLong}\PY{o}{=}\PY{n}{ListePattern}\PY{p}{(}\PY{n}{dfPoleis}\PY{p}{[}\PY{l+s+s1}{\PYZsq{}}\PY{l+s+s1}{Beschreibung}\PY{l+s+s1}{\PYZsq{}}\PY{p}{]}\PY{p}{[}\PY{l+m+mi}{0}\PY{p}{]}\PY{p}{,}\PY{l+s+s2}{\PYZdq{}}\PY{l+s+s2}{(?\PYZlt{}=long}\PY{l+s+s2}{\PYZbs{}}\PY{l+s+s2}{.}\PY{l+s+s2}{\PYZbs{}}\PY{l+s+s2}{s)}\PY{l+s+s2}{\PYZbs{}}\PY{l+s+s2}{s?}\PY{l+s+s2}{\PYZbs{}}\PY{l+s+s2}{d+}\PY{l+s+s2}{\PYZbs{}}\PY{l+s+s2}{.}\PY{l+s+s2}{\PYZbs{}}\PY{l+s+s2}{d+}\PY{l+s+s2}{\PYZdq{}}\PY{p}{)}
         \PY{n}{listLong}
\end{Verbatim}

            \begin{Verbatim}[commandchars=\\\{\}]
{\color{outcolor}Out[{\color{outcolor}19}]:} ['15.05']
\end{Verbatim}
        
    \begin{Verbatim}[commandchars=\\\{\}]
{\color{incolor}In [{\color{incolor}20}]:} \PY{n}{dfPoleis}\PY{p}{[}\PY{l+s+s2}{\PYZdq{}}\PY{l+s+s2}{Beschreibung}\PY{l+s+s2}{\PYZdq{}}\PY{p}{]}\PY{o}{.}\PY{n}{apply}\PY{p}{(}\PY{k}{lambda} \PY{n}{row}\PY{p}{:} \PY{n}{ListePattern}\PY{p}{(}\PY{n}{row}\PY{p}{,}\PY{l+s+s2}{\PYZdq{}}\PY{l+s+s2}{(?\PYZlt{}=Lat}\PY{l+s+s2}{\PYZbs{}}\PY{l+s+s2}{.}\PY{l+s+s2}{\PYZbs{}}\PY{l+s+s2}{s)}\PY{l+s+s2}{\PYZbs{}}\PY{l+s+s2}{s?}\PY{l+s+s2}{\PYZbs{}}\PY{l+s+s2}{d+}\PY{l+s+s2}{\PYZbs{}}\PY{l+s+s2}{.}\PY{l+s+s2}{\PYZbs{}}\PY{l+s+s2}{d+}\PY{l+s+s2}{\PYZdq{}}\PY{p}{)}\PY{p}{)}
\end{Verbatim}

            \begin{Verbatim}[commandchars=\\\{\}]
{\color{outcolor}Out[{\color{outcolor}20}]:} 5. Abakainon         [38.05]
         6. Adranon           [37.40]
         7. Agyrion           [37.40]
         8. Aitna                None
         9. Akragas           [37.20]
         10. Akrai            [37.05]
         11. Alaisa           [38.00]
         12. Alontion         [38.05]
         13. Apollonia        [38.00]
         14. Engyon           [37.45]
         15. Euboia              None
         16. Galeria             None
         17. Gela             [37.05]
         18. Heloron          [36.50]
         19. Henna            [37.35]
         20. Herakleia 1      [37.25]
         21. Herakleia 2         None
         22. Herbes(s)os         None
         23. Herbita             None
         24. Himera           [37.55]
         25. Hippana          [37.40]
         26 *Imachara            None
         27. Kallipolis          None
         28. Kamarina         [36.50]
         29. Kasmenai         [37.05]
         30. Katane           [37.30]
         31. Kentoripa        [37.35]
         32. Kephaloidion     [38.00]
         33. Leontinoi        [37.15]
         34. Lipara           [38.30]
         35. *Longane         [38.05]
         36. Megara           [37.10]
         37. Morgantina       [37.25]
         38. Mylai            [38.15]
         39. Mytistratos      [37.35]
         40. Nakone              None
         41. Naxos            [37.50]
         42. Petra               None
         43. Piakos              None
         44. Selinous         [37.35]
         45. (Sileraioi)         None
         46. (Stielanaioi)    [37.10]
         47. Syrakousai       [37.05]
         48. Tauromenion      [37.50]
         49. Tyndaris         [38.10]
         50. (Tyrrhenoi)         None
         51. Zankle              None
         Name: Beschreibung, dtype: object
\end{Verbatim}
        
    \begin{Verbatim}[commandchars=\\\{\}]
{\color{incolor}In [{\color{incolor}21}]:} \PY{n}{dfPoleis}\PY{p}{[}\PY{l+s+s1}{\PYZsq{}}\PY{l+s+s1}{Latitude}\PY{l+s+s1}{\PYZsq{}}\PY{p}{]} \PY{o}{=} \PY{n}{dfPoleis}\PY{p}{[}\PY{l+s+s1}{\PYZsq{}}\PY{l+s+s1}{Beschreibung}\PY{l+s+s1}{\PYZsq{}}\PY{p}{]}\PY{o}{.}\PY{n}{apply}\PY{p}{(}\PY{k}{lambda} \PY{n}{raw}\PY{p}{:} \PY{n}{ListePattern}\PY{p}{(}\PY{n}{raw}\PY{p}{,}\PY{l+s+s2}{\PYZdq{}}\PY{l+s+s2}{(?\PYZlt{}=Lat}\PY{l+s+s2}{\PYZbs{}}\PY{l+s+s2}{.}\PY{l+s+s2}{\PYZbs{}}\PY{l+s+s2}{s)}\PY{l+s+s2}{\PYZbs{}}\PY{l+s+s2}{d+}\PY{l+s+s2}{\PYZbs{}}\PY{l+s+s2}{.}\PY{l+s+s2}{\PYZbs{}}\PY{l+s+s2}{d+}\PY{l+s+s2}{\PYZdq{}}\PY{p}{)}\PY{p}{)}
         
         \PY{n}{dfPoleis}\PY{p}{[}\PY{l+s+s1}{\PYZsq{}}\PY{l+s+s1}{Longitude}\PY{l+s+s1}{\PYZsq{}}\PY{p}{]} \PY{o}{=} \PY{n}{dfPoleis}\PY{p}{[}\PY{l+s+s1}{\PYZsq{}}\PY{l+s+s1}{Beschreibung}\PY{l+s+s1}{\PYZsq{}}\PY{p}{]}\PY{o}{.}\PY{n}{apply}\PY{p}{(}\PY{k}{lambda} \PY{n}{raw}\PY{p}{:} \PY{n}{ListePattern}\PY{p}{(}\PY{n}{raw}\PY{p}{,}\PY{l+s+s2}{\PYZdq{}}\PY{l+s+s2}{(?\PYZlt{}=long}\PY{l+s+s2}{\PYZbs{}}\PY{l+s+s2}{.}\PY{l+s+s2}{\PYZbs{}}\PY{l+s+s2}{s)}\PY{l+s+s2}{\PYZbs{}}\PY{l+s+s2}{s*}\PY{l+s+s2}{\PYZbs{}}\PY{l+s+s2}{d}\PY{l+s+si}{\PYZob{}2\PYZcb{}}\PY{l+s+s2}{\PYZbs{}}\PY{l+s+s2}{.}\PY{l+s+s2}{\PYZbs{}}\PY{l+s+s2}{d}\PY{l+s+si}{\PYZob{}2\PYZcb{}}\PY{l+s+s2}{\PYZdq{}}\PY{p}{)}\PY{p}{)}
\end{Verbatim}

    \begin{Verbatim}[commandchars=\\\{\}]
{\color{incolor}In [{\color{incolor}22}]:} \PY{n}{dfPoleis}\PY{o}{.}\PY{n}{head}\PY{p}{(}\PY{l+m+mi}{4}\PY{p}{)}
\end{Verbatim}

            \begin{Verbatim}[commandchars=\\\{\}]
{\color{outcolor}Out[{\color{outcolor}22}]:}                                                     Beschreibung       city  \textbackslash{}
         5. Abakainon   (Abakaininos) Map  47.  Lat. 38.05,long. 15.05{\ldots}  bakainon    
         6. Adranon     (Adranites) Map  47.  Lat. 37.40,long.  14.50{\ldots}    dranon    
         7. Agyrion     (Agyrinaios) Map  47.  Lat. 37.40,long.  14.30{\ldots}    gyrion    
         8. Aitna       (Aitnaios) Map  47.Location  of  Aitna  I  as {\ldots}      itna    
         
                        city\_index Latitude Longitude  
         5. Abakainon            5  [38.05]   [15.05]  
         6. Adranon              6  [37.40]  [ 14.50]  
         7. Agyrion              7  [37.40]  [ 14.30]  
         8. Aitna                8     None      None  
\end{Verbatim}
        
    \subsubsection{Zitatnachweise, Namen,
Jahreszahlen}\label{zitatnachweise-namen-jahreszahlen}

    \begin{Verbatim}[commandchars=\\\{\}]
{\color{incolor}In [{\color{incolor}23}]:} \PY{n}{dfPoleis}\PY{p}{[}\PY{l+s+s2}{\PYZdq{}}\PY{l+s+s2}{Beschreibung}\PY{l+s+s2}{\PYZdq{}}\PY{p}{]}\PY{o}{.}\PY{n}{iloc}\PY{p}{[}\PY{l+m+mi}{0}\PY{p}{]} \PY{c+c1}{\PYZsh{} icloc is a method to refer to local index position}
\end{Verbatim}

            \begin{Verbatim}[commandchars=\\\{\}]
{\color{outcolor}Out[{\color{outcolor}23}]:} '(Abakaininos) Map  47.  Lat. 38.05,long. 15.05.  Size  of  territory:  ?  Type:  B: .  The  toponym  is ,  (Diod. 14.90.3)  or  ,  (Diod. 19.65.6;  Steph.  Byz. 2.11).  The  city-ethnic  is   (C 4s  coins,  infra;Diod. 14.78.5;  Steph.  Byz. 2.15). Abakainon  is  called  a  polis  in  the  urban  and  political  sens- es  at  Diod. 14.90.3  (r 393)  and  19.65.6  (r 315),  and  in  the  urban sense  at  14.90.4  (r 393).  The  passage  at  14.90.3  describes  it  as  a polis  symmachis  of  Magon,  and  in  a  later  period  it  was  part  of the  symmachia  of  Agathokles,  alongside  such  poleis  as Kamarina,  Leontinoi,  Katane  and  Messana  (Diod. 19.65.6 (r 315),  19.110.4  (r 311)).  The  internal  collective  use  of  the  city- ethnic  is  found  on  coins  struck  c. 400  (infra),  and  the  exter- nal  collective use  is  found  in  Diod. 19.110.4  (r 311). In 396,  Dionysios  I  deprived  Abakainon  of  a  part  of  its chora,  which  was  handed  over  to  his  new  foundation, Tyndaris  (no. 49;Diod.  14.77.5).  Abakainon  was  situated south-east  of  Tyndaris,  at  modern  Tripi.  The  ancient  city, destroyed  by  the  modern,  is  poorly  known.  However, Diodorus\textbackslash{}'  report  ( 14.90.3)  that  in  393  Carthaginian  troops ²²  For  an  analysis  of  the  conflicts,  see  Manni  ( 1976a)  201\textbackslash{}xad4. 182 fischer-hansen,  nielsen  and  ampolo defeated  by  Dionysios  took  refuge  in  the  city  (  ) suggests that by this date it was fortified.  There are  sporadic  Greek  remains  from  C 6,  and  substantial  Greek remains  from  C 4  (Villard  ( 1954));  the  investigation  of  the extensive  cemetery  north  of  the  city  has  brought  to  light  also monumental  C 4  tombs  of  the  type  known  from  Leontinoi (Bacci  and  Spigo  ( 1997\textbackslash{}xad98));  the  city  minted  a  Greek-style coinage  from  C 5m  (infra). Abakainon  struck  silver  coins  (litra,  hemilitron)  from c.C 5m:  obv.  laureate  head,  bearded  (an  indigenous  god assimilated  to  Zeus)  or  beardless  (assimilated  to  Apollo),  or, on  later  coinage,  female  head  (nymph,  or  Demeter  or Persephone);  rev.  wild  boar  and  acorn,  at  times  a  grain  of barley  or  sow  and  piglet,  legend: ,  (above acorn)  (below), (on  obv.) (on  rev.), (Head,  HN ²  118;  Bertino  ( 1975);  SNG Cop.  Sicily 1\textbackslash{}xad6).  In  C 4s,  the  city  struck  in  bronze:  ( 1) Probably  from  the  time  of  Timoleon:  obv.  female  head;  rev. forepart  of  bull,  or  forepart  of  man-headed  bull,  legend: ,  [],  (Head, HN ²  118;  Bertino  ( 1975)  124\textbackslash{}xad26;  SNG  Cop.  Sicily 7);  ( 2) c. 344\textbackslash{}xad338:  obv.  head  of  Dioskouros,  legend: ;  rev."free horse",  legend: ;  the  obv.  type  may  indicate  influence from  the  mint  of  Tyndaris  or,  more  generally,  from  southern Italy  (Bertino  ( 1975)  124\textbackslash{}xad26).'
\end{Verbatim}
        
    \subsection{Muster (Pattern) zur Erkennung der
Literaturreferenzen}\label{muster-pattern-zur-erkennung-der-literaturreferenzen}

\begin{itemize}
\tightlist
\item
  Primärquellen
\end{itemize}

(Polyb. 1.18.2) (Diod. 13.85.4 (r 406)) (Diod. 13.108.2) (Hdt. 7.165;
IGDS no. 182a) (Pind. Pyth. 6) (Thuc. 6.4.4: µµ ) (Xanthos (FGrHist 765)
fr. 33; Arist. fr. 865)

\begin{itemize}
\tightlist
\item
  Sekundärquellen
\end{itemize}

(Karlsson ( 1995) 161 (Waele ( 1971) 195; Hinz ( 1998) 79)

\begin{itemize}
\tightlist
\item
  Jahreszahlen ( dddd)
\end{itemize}

    \subsubsection{Testen der regulären
Ausdrücke}\label{testen-der-reguluxe4ren-ausdruxfccke}

    Finde alle groß-geschriebenen Wörter mit mindestens 3 nachfolgenden
kleinen Buchstaben.

    \begin{Verbatim}[commandchars=\\\{\}]
{\color{incolor}In [{\color{incolor}24}]:} \PY{n}{dfPoleis}\PY{p}{[}\PY{l+s+s1}{\PYZsq{}}\PY{l+s+s1}{Beschreibung}\PY{l+s+s1}{\PYZsq{}}\PY{p}{]}\PY{o}{.}\PY{n}{apply}\PY{p}{(}\PY{k}{lambda} \PY{n}{raw}\PY{p}{:} \PY{n}{ListePattern}\PY{p}{(}\PY{n}{raw}\PY{p}{,}\PY{l+s+s1}{\PYZsq{}}\PY{l+s+s1}{[A\PYZhy{}Z][a\PYZhy{}z]}\PY{l+s+s1}{\PYZob{}}\PY{l+s+s1}{3,10\PYZcb{}}\PY{l+s+s1}{\PYZsq{}}\PY{p}{)}\PY{p}{)}\PY{p}{[}\PY{l+m+mi}{0}\PY{p}{]}
\end{Verbatim}

            \begin{Verbatim}[commandchars=\\\{\}]
{\color{outcolor}Out[{\color{outcolor}24}]:} ['Abakaininos',
          'Size',
          'Type',
          'Diod',
          'Diod',
          'Steph',
          'Diod',
          'Steph',
          'Abakainon',
          'Diod',
          'Magon',
          'Agathokles',
          'Kamarina',
          'Leontinoi',
          'Katane',
          'Messana',
          'Diod',
          'Diod',
          'Dionysios',
          'Abakainon',
          'Tyndaris',
          'Diod',
          'Abakainon',
          'Tyndaris',
          'Tripi',
          'However',
          'Diodorus',
          'Carthaginia',
          'Manni',
          'Dionysios',
          'There',
          'Greek',
          'Greek',
          'Villard',
          'Leontinoi',
          'Bacci',
          'Spigo',
          'Greek',
          'Abakainon',
          'Zeus',
          'Apollo',
          'Demeter',
          'Persephone',
          'Head',
          'Bertino',
          'Sicily',
          'Probably',
          'Timoleon',
          'Head',
          'Bertino',
          'Sicily',
          'Dioskouros',
          'Tyndaris',
          'Italy',
          'Bertino']
\end{Verbatim}
        
    Finde alle Ausdrücke wie oben, denen ein Punkt folgt, mit anschließenden
Zifferfolgen der Form
{[}Ziffern{]}{[}Punkt{]}{[}Ziffern{]}{[}Punkt{]}{[}Ziffern{]}

    \begin{Verbatim}[commandchars=\\\{\}]
{\color{incolor}In [{\color{incolor}27}]:} \PY{n}{dfPoleis}\PY{p}{[}\PY{l+s+s1}{\PYZsq{}}\PY{l+s+s1}{Beschreibung}\PY{l+s+s1}{\PYZsq{}}\PY{p}{]}\PY{o}{.}\PY{n}{apply}\PY{p}{(}\PY{k}{lambda} \PY{n}{raw}\PY{p}{:} \PY{n}{ListePattern}\PY{p}{(}\PY{n}{raw}\PY{p}{,}\PY{l+s+s1}{\PYZsq{}}\PY{l+s+s1}{[A\PYZhy{}Z][a\PYZhy{}z]}\PY{l+s+s1}{\PYZob{}}\PY{l+s+s1}{1,10\PYZcb{}}\PY{l+s+s1}{\PYZbs{}}\PY{l+s+s1}{. }\PY{l+s+s1}{\PYZbs{}}\PY{l+s+s1}{d}\PY{l+s+s1}{\PYZob{}}\PY{l+s+s1}{1,3\PYZcb{}}\PY{l+s+s1}{\PYZbs{}}\PY{l+s+s1}{.}\PY{l+s+s1}{\PYZbs{}}\PY{l+s+s1}{d}\PY{l+s+s1}{\PYZob{}}\PY{l+s+s1}{1,3\PYZcb{}}\PY{l+s+s1}{\PYZbs{}}\PY{l+s+s1}{.}\PY{l+s+s1}{\PYZbs{}}\PY{l+s+s1}{d}\PY{l+s+s1}{\PYZob{}}\PY{l+s+s1}{1,3\PYZcb{}}\PY{l+s+s1}{\PYZsq{}}\PY{p}{)}\PY{p}{)}\PY{p}{[}\PY{l+m+mi}{0}\PY{p}{]}
\end{Verbatim}

            \begin{Verbatim}[commandchars=\\\{\}]
{\color{outcolor}Out[{\color{outcolor}27}]:} ['Diod. 14.90.3',
          'Diod. 19.65.6',
          'Diod. 14.78.5',
          'Diod. 14.90.3',
          'Diod. 19.65.6',
          'Diod. 19.110.4']
\end{Verbatim}
        
    Finde alle Ausdrücke wie oben, wobei statt des Punktes nach den kleinen
Buchstaben auch zwei Leerzeichen und eine runde Klammer folgen können

    \subsection{Muster zur Erkennung von
Namen}\label{muster-zur-erkennung-von-namen}

Empedokles ( 496) Theron ( 476) Timoleon c. 338

    \begin{Verbatim}[commandchars=\\\{\}]
{\color{incolor}In [{\color{incolor}28}]:} \PY{n}{dfPoleis}\PY{p}{[}\PY{l+s+s1}{\PYZsq{}}\PY{l+s+s1}{Beschreibung}\PY{l+s+s1}{\PYZsq{}}\PY{p}{]}\PY{o}{.}\PY{n}{apply}\PY{p}{(}\PY{k}{lambda} \PY{n}{raw}\PY{p}{:} \PY{n}{ListePattern}\PY{p}{(}\PY{n}{raw}\PY{p}{,}\PY{l+s+s1}{\PYZsq{}}\PY{l+s+s1}{[A\PYZhy{}Z][a\PYZhy{}z]}\PY{l+s+s1}{\PYZob{}}\PY{l+s+s1}{1,10\PYZcb{}[}\PY{l+s+s1}{\PYZbs{}}\PY{l+s+s1}{.| ] [(|}\PY{l+s+s1}{\PYZbs{}}\PY{l+s+s1}{d}\PY{l+s+s1}{\PYZob{}}\PY{l+s+s1}{0,4\PYZcb{}][}\PY{l+s+s1}{\PYZbs{}}\PY{l+s+s1}{d}\PY{l+s+s1}{\PYZob{}}\PY{l+s+s1}{1,4\PYZcb{}| ][}\PY{l+s+s1}{\PYZbs{}}\PY{l+s+s1}{.|}\PY{l+s+s1}{\PYZbs{}}\PY{l+s+s1}{d}\PY{l+s+si}{\PYZob{}\PYZcb{}}\PY{l+s+s1}{]}\PY{l+s+s1}{\PYZbs{}}\PY{l+s+s1}{d}\PY{l+s+s1}{\PYZob{}}\PY{l+s+s1}{1,3\PYZcb{}[}\PY{l+s+s1}{\PYZbs{}}\PY{l+s+s1}{.|}\PY{l+s+s1}{\PYZbs{}}\PY{l+s+s1}{d][}\PY{l+s+s1}{\PYZbs{}}\PY{l+s+s1}{d|)]}\PY{l+s+s1}{\PYZsq{}}\PY{p}{)}\PY{p}{)}\PY{p}{[}\PY{l+m+mi}{0}\PY{p}{]}
\end{Verbatim}

            \begin{Verbatim}[commandchars=\\\{\}]
{\color{outcolor}Out[{\color{outcolor}28}]:} ['Diod. 14.90.3',
          'Diod. 19.65.6',
          'Diod. 14.78.5',
          'Diod. 14.90.3',
          'Diod. 19.65.6',
          'Diod. 19.110.4',
          'Manni  ( 1976',
          'Villard  ( 1954)',
          'Spigo  ( 1997',
          'Bertino  ( 1975)',
          'Bertino  ( 1975)',
          'Bertino  ( 1975)']
\end{Verbatim}
        
    \begin{Verbatim}[commandchars=\\\{\}]
{\color{incolor}In [{\color{incolor}29}]:} \PY{n}{dfPoleis}\PY{p}{[}\PY{l+s+s1}{\PYZsq{}}\PY{l+s+s1}{Namen}\PY{l+s+s1}{\PYZsq{}}\PY{p}{]} \PY{o}{=} \PY{n}{dfPoleis}\PY{p}{[}\PY{l+s+s1}{\PYZsq{}}\PY{l+s+s1}{Beschreibung}\PY{l+s+s1}{\PYZsq{}}\PY{p}{]}\PY{o}{.}\PY{n}{apply}\PY{p}{(}\PY{k}{lambda} \PY{n}{raw}\PY{p}{:} \PY{n}{ListePattern}\PY{p}{(}\PY{n}{raw}\PY{p}{,}\PY{l+s+s1}{\PYZsq{}}\PY{l+s+s1}{[A\PYZhy{}Z][a\PYZhy{}z]}\PY{l+s+s1}{\PYZob{}}\PY{l+s+s1}{3,10\PYZcb{}}\PY{l+s+s1}{\PYZsq{}}\PY{p}{)}\PY{p}{)}
\end{Verbatim}

    \begin{Verbatim}[commandchars=\\\{\}]
{\color{incolor}In [{\color{incolor}30}]:} \PY{n}{dfPoleis}\PY{p}{[}\PY{l+s+s1}{\PYZsq{}}\PY{l+s+s1}{Quellen}\PY{l+s+s1}{\PYZsq{}}\PY{p}{]} \PY{o}{=} \PY{n}{dfPoleis}\PY{p}{[}\PY{l+s+s1}{\PYZsq{}}\PY{l+s+s1}{Beschreibung}\PY{l+s+s1}{\PYZsq{}}\PY{p}{]}\PY{o}{.}\PY{n}{apply}\PY{p}{(}\PY{k}{lambda} \PY{n}{raw}\PY{p}{:} \PY{n}{ListePattern}\PY{p}{(}\PY{n}{raw}\PY{p}{,}\PY{l+s+s1}{\PYZsq{}}\PY{l+s+s1}{[A\PYZhy{}Z][a\PYZhy{}z]}\PY{l+s+s1}{\PYZob{}}\PY{l+s+s1}{1,10\PYZcb{}[}\PY{l+s+s1}{\PYZbs{}}\PY{l+s+s1}{.| ] [(|}\PY{l+s+s1}{\PYZbs{}}\PY{l+s+s1}{d}\PY{l+s+s1}{\PYZob{}}\PY{l+s+s1}{0,4\PYZcb{}][}\PY{l+s+s1}{\PYZbs{}}\PY{l+s+s1}{d}\PY{l+s+s1}{\PYZob{}}\PY{l+s+s1}{1,4\PYZcb{}| ][}\PY{l+s+s1}{\PYZbs{}}\PY{l+s+s1}{.|}\PY{l+s+s1}{\PYZbs{}}\PY{l+s+s1}{d}\PY{l+s+si}{\PYZob{}\PYZcb{}}\PY{l+s+s1}{]}\PY{l+s+s1}{\PYZbs{}}\PY{l+s+s1}{d}\PY{l+s+s1}{\PYZob{}}\PY{l+s+s1}{1,3\PYZcb{}[}\PY{l+s+s1}{\PYZbs{}}\PY{l+s+s1}{.|}\PY{l+s+s1}{\PYZbs{}}\PY{l+s+s1}{d][}\PY{l+s+s1}{\PYZbs{}}\PY{l+s+s1}{d|)]}\PY{l+s+s1}{\PYZsq{}}\PY{p}{)}\PY{p}{)}
\end{Verbatim}

    \begin{Verbatim}[commandchars=\\\{\}]
{\color{incolor}In [{\color{incolor}31}]:} \PY{n}{dfPoleis}\PY{o}{.}\PY{n}{head}\PY{p}{(}\PY{l+m+mi}{4}\PY{p}{)}
\end{Verbatim}

            \begin{Verbatim}[commandchars=\\\{\}]
{\color{outcolor}Out[{\color{outcolor}31}]:}                                                     Beschreibung       city  \textbackslash{}
         5. Abakainon   (Abakaininos) Map  47.  Lat. 38.05,long. 15.05{\ldots}  bakainon    
         6. Adranon     (Adranites) Map  47.  Lat. 37.40,long.  14.50{\ldots}    dranon    
         7. Agyrion     (Agyrinaios) Map  47.  Lat. 37.40,long.  14.30{\ldots}    gyrion    
         8. Aitna       (Aitnaios) Map  47.Location  of  Aitna  I  as {\ldots}      itna    
         
                        city\_index Latitude Longitude  \textbackslash{}
         5. Abakainon            5  [38.05]   [15.05]   
         6. Adranon              6  [37.40]  [ 14.50]   
         7. Agyrion              7  [37.40]  [ 14.30]   
         8. Aitna                8     None      None   
         
                                                                    Namen  \textbackslash{}
         5. Abakainon   [Abakaininos, Size, Type, Diod, Diod, Steph, D{\ldots}   
         6. Adranon     [Adranites, Size, Type, Diod, Steph, Diod, Adr{\ldots}   
         7. Agyrion     [Agyrinaios, Size, Type, Diod, Ptol, Geog, Ste{\ldots}   
         8. Aitna       [Aitnaios, Location, Aitna, Katane, Aitna, Dio{\ldots}   
         
                                                                  Quellen  
         5. Abakainon   [Diod. 14.90.3, Diod. 19.65.6, Diod. 14.78.5, {\ldots}  
         6. Adranon     [Diod. 14.37.5, Diod. 16.68.9, Diod. 14.37.5, {\ldots}  
         7. Agyrion     [Byz. 23.19), Diod. 16.82.4, Moggi  ( 1976), D{\ldots}  
         8. Aitna       [Diod. 11.49.1, Diod. 11.49.1, Diod. 11.66.4, {\ldots}  
\end{Verbatim}
        
    \section{Datenvalidierung}\label{datenvalidierung}

    \subsection{Bewertung des Modells mit Performanz-
(Konfusions-)matrix}\label{bewertung-des-modells-mit-performanz--konfusions-matrix}

    Diskussion der Performanzmatrix: vier Fälle - soll match
vs.~tatsächlicher match - nicht soll match vs.~tatsächlich - soll match
vs.~nicht tatsächlich - nicht soll vs.~nicht tatsächlich

    \subsection{Wertverteilungen, Test auf
Dopplungen}\label{wertverteilungen-test-auf-dopplungen}

    Lese Werte der Spalte Quellen als Liste aus.

    \begin{Verbatim}[commandchars=\\\{\}]
{\color{incolor}In [{\color{incolor}32}]:} \PY{n}{mainList} \PY{o}{=} \PY{n}{dfPoleis}\PY{p}{[}\PY{l+s+s1}{\PYZsq{}}\PY{l+s+s1}{Quellen}\PY{l+s+s1}{\PYZsq{}}\PY{p}{]}\PY{o}{.}\PY{n}{values}\PY{o}{.}\PY{n}{tolist}\PY{p}{(}\PY{p}{)}
         \PY{n}{mainList}\PY{p}{[}\PY{l+m+mi}{0}\PY{p}{]}
\end{Verbatim}

            \begin{Verbatim}[commandchars=\\\{\}]
{\color{outcolor}Out[{\color{outcolor}32}]:} ['Diod. 14.90.3',
          'Diod. 19.65.6',
          'Diod. 14.78.5',
          'Diod. 14.90.3',
          'Diod. 19.65.6',
          'Diod. 19.110.4',
          'Manni  ( 1976',
          'Villard  ( 1954)',
          'Spigo  ( 1997',
          'Bertino  ( 1975)',
          'Bertino  ( 1975)',
          'Bertino  ( 1975)']
\end{Verbatim}
        
    Reduziere Unterlisten auf eine Gesamtliste.

    \begin{Verbatim}[commandchars=\\\{\}]
{\color{incolor}In [{\color{incolor}50}]:} \PY{n}{quellenListe} \PY{o}{=} \PY{p}{[}\PY{p}{]}
         \PY{k}{for} \PY{n}{sublist} \PY{o+ow}{in} \PY{n}{mainList}\PY{p}{:}
             \PY{k}{if} \PY{n}{sublist}\PY{p}{:}
                 \PY{k}{for} \PY{n}{k} \PY{o+ow}{in} \PY{n}{sublist}\PY{p}{:}
                     \PY{n}{quellenListe}\PY{o}{.}\PY{n}{append}\PY{p}{(}\PY{n}{k}\PY{p}{)}
\end{Verbatim}

    \begin{Verbatim}[commandchars=\\\{\}]
{\color{incolor}In [{\color{incolor}51}]:} \PY{n}{quellenListe}\PY{p}{[}\PY{p}{:}\PY{l+m+mi}{10}\PY{p}{]}
\end{Verbatim}

            \begin{Verbatim}[commandchars=\\\{\}]
{\color{outcolor}Out[{\color{outcolor}51}]:} ['Diod. 14.90.3',
          'Diod. 19.65.6',
          'Diod. 14.78.5',
          'Diod. 14.90.3',
          'Diod. 19.65.6',
          'Diod. 19.110.4',
          'Manni  ( 1976',
          'Villard  ( 1954)',
          'Spigo  ( 1997',
          'Bertino  ( 1975)']
\end{Verbatim}
        
    Zähle die Häufigkeit der verschiedenen Quellen und speichere als
Dictionary.

    \begin{Verbatim}[commandchars=\\\{\}]
{\color{incolor}In [{\color{incolor}52}]:} \PY{n}{quellenVerteilung} \PY{o}{=} \PY{p}{\PYZob{}}\PY{n}{x}\PY{p}{:}\PY{n}{quellenListe}\PY{o}{.}\PY{n}{count}\PY{p}{(}\PY{n}{x}\PY{p}{)} \PY{k}{for} \PY{n}{x} \PY{o+ow}{in} \PY{n}{quellenListe}\PY{p}{\PYZcb{}}
         \PY{n}{quellenVerteilung}\PY{p}{[}\PY{l+s+s1}{\PYZsq{}}\PY{l+s+s1}{Diod. 14.90.3}\PY{l+s+s1}{\PYZsq{}}\PY{p}{]}
\end{Verbatim}

            \begin{Verbatim}[commandchars=\\\{\}]
{\color{outcolor}Out[{\color{outcolor}52}]:} 2
\end{Verbatim}
        
    Erzeuge DataFrame, mit neuem Index und Namen der Spalten. Sortiere
diesen Nach der Häufigkeit der Quelle.

    \begin{Verbatim}[commandchars=\\\{\}]
{\color{incolor}In [{\color{incolor}88}]:} \PY{n}{dfQuellenVerteilung} \PY{o}{=} \PY{n}{pd}\PY{o}{.}\PY{n}{DataFrame}\PY{p}{(}\PY{p}{[}\PY{n}{quellenVerteilung}\PY{p}{]}\PY{p}{)}
         \PY{n}{dfQuellenVerteilung}
\end{Verbatim}

            \begin{Verbatim}[commandchars=\\\{\}]
{\color{outcolor}Out[{\color{outcolor}88}]:}    Adamesteanu  ( 1986)  Adamesteanu  ( 1994  Agata  ( 1989)  Albini  ( 1964)  \textbackslash{}
         0                     1                    1               2                1   
         
            Allegro  ( 1991)  Allegro  ( 1997)  Angelis  ( 1994)  Anti  ( 1947)  \textbackslash{}
         0                 2                 1                 2              1   
         
            Anti  ( 1981)  Antonaccio  ( 1997)        {\ldots}         Westermark  ( 1998)  \textbackslash{}
         0              3                    4        {\ldots}                           2   
         
            White  ( 1964)  Wilson  ( 1988)  Wilson  ( 1996)  Winter  ( 1963)  \textbackslash{}
         0               1                1                1                1   
         
            Winter  ( 1971)  Yalouris  ( 1980)  Zahrnt  ( 1993)  Ziegler  ( 1934)  \textbackslash{}
         0                1                  2                2                 3   
         
            Ziegler  ( 1943)  
         0                 1  
         
         [1 rows x 567 columns]
\end{Verbatim}
        
    \begin{Verbatim}[commandchars=\\\{\}]
{\color{incolor}In [{\color{incolor}89}]:} \PY{n}{dfQuellenVerteilung} \PY{o}{=} \PY{n}{dfQuellenVerteilung}\PY{o}{.}\PY{n}{transpose}\PY{p}{(}\PY{p}{)}\PY{o}{.}\PY{n}{reset\PYZus{}index}\PY{p}{(}\PY{p}{)}
         \PY{n}{dfQuellenVerteilung}\PY{o}{.}\PY{n}{head}\PY{p}{(}\PY{p}{)}
\end{Verbatim}

            \begin{Verbatim}[commandchars=\\\{\}]
{\color{outcolor}Out[{\color{outcolor}89}]:}                   index  0
         0  Adamesteanu  ( 1986)  1
         1   Adamesteanu  ( 1994  1
         2        Agata  ( 1989)  2
         3       Albini  ( 1964)  1
         4      Allegro  ( 1991)  2
\end{Verbatim}
        
    \begin{Verbatim}[commandchars=\\\{\}]
{\color{incolor}In [{\color{incolor}90}]:} \PY{n}{dfQuellenVerteilung} \PY{o}{=} \PY{n}{dfQuellenVerteilung}\PY{o}{.}\PY{n}{rename}\PY{p}{(}\PY{n}{columns}\PY{o}{=}\PY{p}{\PYZob{}}\PY{l+s+s1}{\PYZsq{}}\PY{l+s+s1}{index}\PY{l+s+s1}{\PYZsq{}}\PY{p}{:} \PY{l+s+s1}{\PYZsq{}}\PY{l+s+s1}{Quelle}\PY{l+s+s1}{\PYZsq{}}\PY{p}{,} \PY{l+m+mi}{0}\PY{p}{:}\PY{l+s+s1}{\PYZsq{}}\PY{l+s+s1}{Häufigkeit}\PY{l+s+s1}{\PYZsq{}}\PY{p}{\PYZcb{}}\PY{p}{)}
         \PY{n}{dfQuellenVerteilung}\PY{o}{.}\PY{n}{head}\PY{p}{(}\PY{p}{)}
\end{Verbatim}

            \begin{Verbatim}[commandchars=\\\{\}]
{\color{outcolor}Out[{\color{outcolor}90}]:}                  Quelle  Häufigkeit
         0  Adamesteanu  ( 1986)           1
         1   Adamesteanu  ( 1994           1
         2        Agata  ( 1989)           2
         3       Albini  ( 1964)           1
         4      Allegro  ( 1991)           2
\end{Verbatim}
        
    \begin{Verbatim}[commandchars=\\\{\}]
{\color{incolor}In [{\color{incolor}56}]:} \PY{n}{dfQuellenVerteilung}\PY{o}{.}\PY{n}{sort\PYZus{}values}\PY{p}{(}\PY{n}{by}\PY{o}{=}\PY{l+s+s1}{\PYZsq{}}\PY{l+s+s1}{Häufigkeit}\PY{l+s+s1}{\PYZsq{}}\PY{p}{,}\PY{n}{ascending}\PY{o}{=}\PY{k+kc}{False}\PY{p}{)}\PY{o}{.}\PY{n}{head}\PY{p}{(}\PY{l+m+mi}{10}\PY{p}{)}
\end{Verbatim}

            \begin{Verbatim}[commandchars=\\\{\}]
{\color{outcolor}Out[{\color{outcolor}56}]:}                   Quelle  Häufigkeit
         382    Manganaro  ( 1996          15
         331        Hinz  ( 1998)          13
         509     Talbert  ( 1974)          13
         83     Cavalier  ( 1991)          11
         344    Karlsson  ( 1995)           9
         226        Diod. 14.78.7           9
         113        Diod. 11.49.2           9
         47   Boehringer  ( 1998)           8
         478      Rutter  ( 1997)           8
         327      Hansen  ( 2000)           8
\end{Verbatim}
        
    \section{Größere Textblöcke, mit textblob and
NLTK}\label{gruxf6uxdfere-textbluxf6cke-mit-textblob-and-nltk}

    \begin{Verbatim}[commandchars=\\\{\}]
{\color{incolor}In [{\color{incolor}34}]:} \PY{k+kn}{from} \PY{n+nn}{textblob} \PY{k}{import} \PY{n}{TextBlob}
         \PY{k+kn}{from} \PY{n+nn}{textblob}\PY{n+nn}{.}\PY{n+nn}{taggers} \PY{k}{import} \PY{n}{NLTKTagger}
         
         \PY{k+kn}{from} \PY{n+nn}{nltk}\PY{n+nn}{.}\PY{n+nn}{tokenize} \PY{k}{import} \PY{n}{SExprTokenizer}
         \PY{n}{nltk\PYZus{}tagger} \PY{o}{=} \PY{n}{NLTKTagger}\PY{p}{(}\PY{p}{)}
         
         \PY{k+kn}{import} \PY{n+nn}{json}
         \PY{k+kn}{import} \PY{n+nn}{pandas} \PY{k}{as} \PY{n+nn}{pd}
         \PY{k+kn}{import} \PY{n+nn}{re}
             
         \PY{k+kn}{import} \PY{n+nn}{nltk}\PY{n+nn}{.}\PY{n+nn}{data}
         \PY{k+kn}{from} \PY{n+nn}{nltk} \PY{k}{import} \PY{n}{PunktSentenceTokenizer}
\end{Verbatim}

    \subsection{Alternative loading of data from online
resources}\label{alternative-loading-of-data-from-online-resources}

Laden der gesamt Poleis Quell-Datei

    \begin{Verbatim}[commandchars=\\\{\}]
{\color{incolor}In [{\color{incolor}35}]:} \PY{k+kn}{import} \PY{n+nn}{requests} \PY{c+c1}{\PYZsh{} To communicate with websites}
\end{Verbatim}

    \subsubsection{Github}\label{github}

    \begin{Verbatim}[commandchars=\\\{\}]
{\color{incolor}In [{\color{incolor}36}]:} \PY{c+c1}{\PYZsh{} The Github site for the lecture is public, therefore we can access the data in the following manner: }
         \PY{n}{PoleisDataOnline} \PY{o}{=} \PY{n}{requests}\PY{o}{.}\PY{n}{get}\PY{p}{(}\PY{l+s+s1}{\PYZsq{}}\PY{l+s+s1}{https://raw.githubusercontent.com/computational\PYZhy{}humanities/vorlesung/master/Notebooks/secondVersion.json}\PY{l+s+s1}{\PYZsq{}}\PY{p}{)}
         \PY{n}{PoleisRawData2} \PY{o}{=} \PY{n}{PoleisDataOnline}\PY{o}{.}\PY{n}{json}\PY{p}{(}\PY{p}{)}
         \PY{n}{PoleisRawData2}\PY{o}{.}\PY{n}{keys}\PY{p}{(}\PY{p}{)}
\end{Verbatim}

            \begin{Verbatim}[commandchars=\\\{\}]
{\color{outcolor}Out[{\color{outcolor}36}]:} dict\_keys(['Ionia', 'Doris', 'Rhodos', 'Troas', 'Italia and Kampania', 'The Propontic Coast of Asia Minor', 'The Aegean', 'Karia', 'Lesbos', 'Spain and France (including Corsica)', 'The Black Sea Area', 'The Adriatic', 'The Saronic Gulf', 'Boiotia', 'Thrace from Axios to Strymon', 'Achaia', 'Euboia', 'Megaris, Korinthia, Sikyonia', 'Thracian Chersonesos', 'Propontic Thrace', 'Lykia', 'West Lokris', 'Thrace from Strymon to Nestos', 'Inland Thrace', 'Attika', 'Elis', 'Argolis', 'Crete', 'Aiolis and South-western Mysia', 'Aitolia', 'Thrace from Nestos to Hebros', 'Epeiros', 'Cyprus', 'Sikelia', 'Triphylia', 'Thessalia and Adjacent Regions', 'East Lokris', 'Lakedaimon', 'Arkadia', 'Makedonia', 'Phokis', 'Messenia', 'The South Coast of Asia Minor (Pamphylia Kilikia)', 'Akarnania and Adjacent Areas'])
\end{Verbatim}
        
    \subsubsection{edition-topoi.org}\label{edition-topoi.org}

    \begin{Verbatim}[commandchars=\\\{\}]
{\color{incolor}In [{\color{incolor}37}]:} \PY{n}{PoleisDataOnline2} \PY{o}{=} \PY{n}{requests}\PY{o}{.}\PY{n}{get}\PY{p}{(}\PY{l+s+s1}{\PYZsq{}}\PY{l+s+s1}{http://repository.edition\PYZhy{}topoi.org/MISC/ReposMISC/MISC00005/secondVersion.json}\PY{l+s+s1}{\PYZsq{}}\PY{p}{)}
         \PY{n}{PoleisRawData3} \PY{o}{=} \PY{n}{PoleisDataOnline2}\PY{o}{.}\PY{n}{json}\PY{p}{(}\PY{p}{)}
         \PY{n}{PoleisRawData3}\PY{o}{.}\PY{n}{keys}\PY{p}{(}\PY{p}{)}
\end{Verbatim}

            \begin{Verbatim}[commandchars=\\\{\}]
{\color{outcolor}Out[{\color{outcolor}37}]:} dict\_keys(['Ionia', 'Doris', 'Rhodos', 'Troas', 'Italia and Kampania', 'The Propontic Coast of Asia Minor', 'The Aegean', 'Karia', 'Lesbos', 'Spain and France (including Corsica)', 'The Black Sea Area', 'The Adriatic', 'The Saronic Gulf', 'Boiotia', 'Thrace from Axios to Strymon', 'Achaia', 'Euboia', 'Megaris, Korinthia, Sikyonia', 'Thracian Chersonesos', 'Propontic Thrace', 'Lykia', 'West Lokris', 'Thrace from Strymon to Nestos', 'Inland Thrace', 'Attika', 'Elis', 'Argolis', 'Crete', 'Aiolis and South-western Mysia', 'Aitolia', 'Thrace from Nestos to Hebros', 'Epeiros', 'Cyprus', 'Sikelia', 'Triphylia', 'Thessalia and Adjacent Regions', 'East Lokris', 'Lakedaimon', 'Arkadia', 'Makedonia', 'Phokis', 'Messenia', 'The South Coast of Asia Minor (Pamphylia Kilikia)', 'Akarnania and Adjacent Areas'])
\end{Verbatim}
        
    \subsection{Eigenschaften des Datasets
PoleisRawData2}\label{eigenschaften-des-datasets-poleisrawdata2}

    Each region contains the city names as second-level keys

    \begin{Verbatim}[commandchars=\\\{\}]
{\color{incolor}In [{\color{incolor}38}]:} \PY{n}{PoleisRawData2}\PY{p}{[}\PY{l+s+s1}{\PYZsq{}}\PY{l+s+s1}{Karia}\PY{l+s+s1}{\PYZsq{}}\PY{p}{]}\PY{o}{.}\PY{n}{keys}\PY{p}{(}\PY{p}{)}
\end{Verbatim}

            \begin{Verbatim}[commandchars=\\\{\}]
{\color{outcolor}Out[{\color{outcolor}38}]:} dict\_keys(['Salmakis', 'Bolbai', 'Bargasa', 'Telemessos', 'Olymos', 'Telandros', 'Hydisos', 'Thydonos', 'Halikarnassos', 'Pidasa', 'Alabanda', 'Pladasa', 'Kasolaba', 'Passanda', 'Medmasos', 'Kalynda', 'Alinda', 'Naryandos', 'Aulai', 'Ouranion', 'Naxia', 'Kyrbissos', 'Amos', 'Myndos', 'Taramptos', 'Iasos', 'Euromos', 'Pedasa', 'Kindye', 'with', 'Lepsimandos', 'Chersonesos', 'Kyllandos', 'Arlissos', 'Idrias', 'Pyrnos', 'Pyrindos', 'Bargylia', 'Chios', 'Mylasa', 'Keramos', 'Kaunos', 'Knidos', 'Tralleis'])
\end{Verbatim}
        
    To access the text for a certain city, one has to use first and second
level keys

    \begin{Verbatim}[commandchars=\\\{\}]
{\color{incolor}In [{\color{incolor}39}]:} \PY{n}{hydisosText} \PY{o}{=} \PY{n}{PoleisRawData2}\PY{p}{[}\PY{l+s+s1}{\PYZsq{}}\PY{l+s+s1}{Karia}\PY{l+s+s1}{\PYZsq{}}\PY{p}{]}\PY{p}{[}\PY{l+s+s1}{\PYZsq{}}\PY{l+s+s1}{Hydisos}\PY{l+s+s1}{\PYZsq{}}\PY{p}{]}
\end{Verbatim}

    \begin{Verbatim}[commandchars=\\\{\}]
{\color{incolor}In [{\color{incolor}40}]:} \PY{n}{hydisosText}
\end{Verbatim}

            \begin{Verbatim}[commandchars=\\\{\}]
{\color{outcolor}Out[{\color{outcolor}40}]:} 'Identifier: 891. , (Hydisseus) Map  61.  Lat. 37.10,long.  27.50. Size  of  territory:  ?  Type:  B:?  The  toponym  is  ` (Steph.  Byz. 645.17).  The  earliest  attestation  of  the  toponym is  in  a  C 1  inscription  (I.Stratonikeia 508.10  (c. 81)): `, although  it  was  mentioned  by  Apollonius  Aphrodisiensis, whose may  be  dated  to  C 3  (FGrHist 740  fr. 4).  The city-ethnic  is  `  (IG i³  265.ii.51;Apollonius Aphrodisiensis  (FGrHist 740)  fr. 4  (perhaps  C 3))  or `  (I.Mylasa 401.8  (C 2\textbackslash{}xadC1)). Hydisos  was  a  member  of  the  Delian  League,  but  is  registered  only  twice,  in  448/7  (IG i³  264.iii.21,  restored: `[\textasciitilde{}]) and 447/6  (IG i³  265.ii.51,  restored: `[\textasciitilde{}]),  paying  a  phoros  of 1  tal. At  the  site  of  Hydisos  there  are  remains  of  city  walls  and towers,  probably  of  early  Hellenistic  date  (L.  Robert  ( 1935) 339\textbackslash{}xad40). 890.  (Hymisseis) Map 61,  unlocated,  but  possibly  situat- ed  between  Amyzon  (no. 874)  and  Mylasa  (no. 913)  (Pontani ( 1997)  7;  cf.  L.  Robert  ( 1955)  226).  Type:  C: .  A  toponym  is not  attested.  The  city-ethnic  is  `µ  (IG i³  262.iv.19; restored:  [h ]µ\textasciitilde{}) or `µ  (IG i³  71.ii.143;  IG xii suppl. 127.58  (C 3)).  The  collective  use  of  the  city-ethnic  is attested  externally  in  the  tribute  lists  and  in  the  assessment decree  of 425/4  (IG i³  71.ii.143). The  Hymisseis  were  members  of  the  Delian  League.  They are  recorded  three  times  in  the  lists  from  451/0  (IG i³ 262.iv.19)  to  447/6  (IG i³  265.ii.50),  paying  a  phoros  of 1,200 dr.  In  the  assessment  decree  of 425/4  they  form  a  syntely  with the  Edrieis  (no. 892)  and  Kyromeis,  and  the  three  together are  assessed  at  6  tal.  (IG i³  71.ii.143\textbackslash{}xad44).  A  Hymesseus  is recorded  in a C 3  list  of  proxenoi  from  Eresos  (no. 796)  (IG xii suppl. 127.58). 891. '
\end{Verbatim}
        
    \subsubsection{Textblob}\label{textblob}

Generate blob using TextBlob. This allows identifying different parts of
a text (words, sentences) or tagging words with their type
(noun,verb,etc)

    \begin{Verbatim}[commandchars=\\\{\}]
{\color{incolor}In [{\color{incolor}41}]:} \PY{n}{blob} \PY{o}{=} \PY{n}{TextBlob}\PY{p}{(}\PY{n}{hydisosText}\PY{p}{)}
\end{Verbatim}

    Return noun phrases

    \begin{Verbatim}[commandchars=\\\{\}]
{\color{incolor}In [{\color{incolor}45}]:} \PY{n}{blob}\PY{o}{.}\PY{n}{noun\PYZus{}phrases}
\end{Verbatim}

            \begin{Verbatim}[commandchars=\\\{\}]
{\color{outcolor}Out[{\color{outcolor}45}]:} WordList(['identifier', 'hydisseus', 'lat', 'size', 'type', 'steph', 'byz', 'i.stratonikeia', 'apollonius aphrodisiensis', 'fgrhist', 'ig', 'i³ 265.ii.51', 'apollonius aphrodisiensis', 'fgrhist', 'i.mylasa', 'c 2\textbackslash{}xadc1', 'hydisos', 'delian', 'ig', 'i³ 264.iii.21', '` [ \textasciitilde{} ]', 'ig', 'i³ 265.ii.51', '` [ \textasciitilde{} ]', 'hydisos', 'city walls', 'hellenistic', 'l. robert', 'hymisseis', 'amyzon', 'mylasa', 'pontani', 'l. robert', 'type', 'ig', 'i³ 262.iv.19', '[ h ] µ\textasciitilde{}', 'ig', 'i³ 71.ii.143', 'ig', 'xii suppl', 'collective use', 'assessment decree', 'ig', 'i³ 71.ii.143', 'hymisseis', 'delian', 'ig', 'i³ 262.iv.19', 'ig', 'i³ 265.ii.50', 'assessment decree', 'edrieis', 'kyromeis', 'ig', 'i³ 71.ii.143\textbackslash{}xad44', 'hymesseus', 'eresos', 'ig', 'xii suppl'])
\end{Verbatim}
        
    Return tags of first 10 words, for definition see
e.g.~https://www.ling.upenn.edu/courses/Fall\_2003/ling001/penn\_treebank\_pos.html

    \begin{Verbatim}[commandchars=\\\{\}]
{\color{incolor}In [{\color{incolor}46}]:} \PY{n}{blob}\PY{o}{.}\PY{n}{tags}\PY{p}{[}\PY{p}{:}\PY{l+m+mi}{10}\PY{p}{]}
\end{Verbatim}

            \begin{Verbatim}[commandchars=\\\{\}]
{\color{outcolor}Out[{\color{outcolor}46}]:} [('Identifier', 'NN'),
          ('891.', 'CD'),
          ('Hydisseus', 'NNP'),
          ('Map', 'NNP'),
          ('61', 'CD'),
          ('Lat', 'NNP'),
          ('37.10', 'CD'),
          ('long', 'RB'),
          ('27.50', 'CD'),
          ('Size', 'NN')]
\end{Verbatim}
        
    Return list of words

    \begin{Verbatim}[commandchars=\\\{\}]
{\color{incolor}In [{\color{incolor}48}]:} \PY{n}{blob}\PY{o}{.}\PY{n}{words}
\end{Verbatim}

            \begin{Verbatim}[commandchars=\\\{\}]
{\color{outcolor}Out[{\color{outcolor}48}]:} WordList(['Identifier', '891', 'Hydisseus', 'Map', '61', 'Lat', '37.10', 'long', '27.50', 'Size', 'of', 'territory', 'Type', 'B', 'The', 'toponym', 'is', 'Steph', 'Byz', '645.17', 'The', 'earliest', 'attestation', 'of', 'the', 'toponym', 'is', 'in', 'a', 'C', '1', 'inscription', 'I.Stratonikeia', '508.10', 'c', '81', 'although', 'it', 'was', 'mentioned', 'by', 'Apollonius', 'Aphrodisiensis', 'whose', 'may', 'be', 'dated', 'to', 'C', '3', 'FGrHist', '740', 'fr', '4', 'The', 'city-ethnic', 'is', 'IG', 'i³', '265.ii.51', 'Apollonius', 'Aphrodisiensis', 'FGrHist', '740', 'fr', '4', 'perhaps', 'C', '3', 'or', 'I.Mylasa', '401.8', 'C', '2\textbackslash{}xadC1', 'Hydisos', 'was', 'a', 'member', 'of', 'the', 'Delian', 'League', 'but', 'is', 'registered', 'only', 'twice', 'in', '448/7', 'IG', 'i³', '264.iii.21', 'restored', 'and', '447/6', 'IG', 'i³', '265.ii.51', 'restored', 'paying', 'a', 'phoros', 'of', '1', 'tal', 'At', 'the', 'site', 'of', 'Hydisos', 'there', 'are', 'remains', 'of', 'city', 'walls', 'and', 'towers', 'probably', 'of', 'early', 'Hellenistic', 'date', 'L', 'Robert', '1935', '339\textbackslash{}xad40', '890', 'Hymisseis', 'Map', '61', 'unlocated', 'but', 'possibly', 'situat', 'ed', 'between', 'Amyzon', 'no', '874', 'and', 'Mylasa', 'no', '913', 'Pontani', '1997', '7', 'cf', 'L', 'Robert', '1955', '226', 'Type', 'C', 'A', 'toponym', 'is', 'not', 'attested', 'The', 'city-ethnic', 'is', 'µ', 'IG', 'i³', '262.iv.19', 'restored', 'h', 'µ', 'or', 'µ', 'IG', 'i³', '71.ii.143', 'IG', 'xii', 'suppl', '127.58', 'C', '3', 'The', 'collective', 'use', 'of', 'the', 'city-ethnic', 'is', 'attested', 'externally', 'in', 'the', 'tribute', 'lists', 'and', 'in', 'the', 'assessment', 'decree', 'of', '425/4', 'IG', 'i³', '71.ii.143', 'The', 'Hymisseis', 'were', 'members', 'of', 'the', 'Delian', 'League', 'They', 'are', 'recorded', 'three', 'times', 'in', 'the', 'lists', 'from', '451/0', 'IG', 'i³', '262.iv.19', 'to', '447/6', 'IG', 'i³', '265.ii.50', 'paying', 'a', 'phoros', 'of', '1,200', 'dr', 'In', 'the', 'assessment', 'decree', 'of', '425/4', 'they', 'form', 'a', 'syntely', 'with', 'the', 'Edrieis', 'no', '892', 'and', 'Kyromeis', 'and', 'the', 'three', 'together', 'are', 'assessed', 'at', '6', 'tal', 'IG', 'i³', '71.ii.143\textbackslash{}xad44', 'A', 'Hymesseus', 'is', 'recorded', 'in', 'a', 'C', '3', 'list', 'of', 'proxenoi', 'from', 'Eresos', 'no', '796', 'IG', 'xii', 'suppl', '127.58', '891'])
\end{Verbatim}
        
    Return list of sentences

    \begin{Verbatim}[commandchars=\\\{\}]
{\color{incolor}In [{\color{incolor}47}]:} \PY{n}{blob}\PY{o}{.}\PY{n}{sentences}
\end{Verbatim}

            \begin{Verbatim}[commandchars=\\\{\}]
{\color{outcolor}Out[{\color{outcolor}47}]:} [Sentence("Identifier: 891. , (Hydisseus) Map  61."),
          Sentence("Lat."),
          Sentence("37.10,long."),
          Sentence("27.50."),
          Sentence("Size  of  territory:  ?"),
          Sentence("Type:  B:?"),
          Sentence("The  toponym  is  ` (Steph."),
          Sentence("Byz."),
          Sentence("645.17)."),
          Sentence("The  earliest  attestation  of  the  toponym is  in  a  C 1  inscription  (I.Stratonikeia 508.10  (c. 81)): `, although  it  was  mentioned  by  Apollonius  Aphrodisiensis, whose may  be  dated  to  C 3  (FGrHist 740  fr."),
          Sentence("4)."),
          Sentence("The city-ethnic  is  `  (IG i³  265.ii.51;Apollonius Aphrodisiensis  (FGrHist 740)  fr."),
          Sentence("4  (perhaps  C 3))  or `  (I.Mylasa 401.8  (C 2­C1))."),
          Sentence("Hydisos  was  a  member  of  the  Delian  League,  but  is  registered  only  twice,  in  448/7  (IG i³  264.iii.21,  restored: `[\textasciitilde{}]) and 447/6  (IG i³  265.ii.51,  restored: `[\textasciitilde{}]),  paying  a  phoros  of 1  tal."),
          Sentence("At  the  site  of  Hydisos  there  are  remains  of  city  walls  and towers,  probably  of  early  Hellenistic  date  (L.  Robert  ( 1935) 339­40)."),
          Sentence("890."),
          Sentence("(Hymisseis) Map 61,  unlocated,  but  possibly  situat- ed  between  Amyzon  (no."),
          Sentence("874)  and  Mylasa  (no."),
          Sentence("913)  (Pontani ( 1997)  7;  cf."),
          Sentence("L.  Robert  ( 1955)  226)."),
          Sentence("Type:  C: ."),
          Sentence("A  toponym  is not  attested."),
          Sentence("The  city-ethnic  is  `µ  (IG i³  262.iv.19; restored:  [h ]µ\textasciitilde{}) or `µ  (IG i³  71.ii.143;  IG xii suppl."),
          Sentence("127.58  (C 3))."),
          Sentence("The  collective  use  of  the  city-ethnic  is attested  externally  in  the  tribute  lists  and  in  the  assessment decree  of 425/4  (IG i³  71.ii.143)."),
          Sentence("The  Hymisseis  were  members  of  the  Delian  League."),
          Sentence("They are  recorded  three  times  in  the  lists  from  451/0  (IG i³ 262.iv.19)  to  447/6  (IG i³  265.ii.50),  paying  a  phoros  of 1,200 dr."),
          Sentence("In  the  assessment  decree  of 425/4  they  form  a  syntely  with the  Edrieis  (no."),
          Sentence("892)  and  Kyromeis,  and  the  three  together are  assessed  at  6  tal."),
          Sentence("(IG i³  71.ii.143­44)."),
          Sentence("A  Hymesseus  is recorded  in a C 3  list  of  proxenoi  from  Eresos  (no."),
          Sentence("796)  (IG xii suppl."),
          Sentence("127.58)."),
          Sentence("891. ")]
\end{Verbatim}
        
    Not all sentences are correct. We have to use a different tokenizer to
improve this.

By calling PunktSentenceTokenizer with an input text, we can train the
detection of sentences.

This is usually a problem, since a lot of citations (parenthesis) or
special characters hinder the detection of a sentence end.

    To generate a text of all cities in a region we can use

    \begin{Verbatim}[commandchars=\\\{\}]
{\color{incolor}In [{\color{incolor}52}]:} \PY{n}{ioniaText} \PY{o}{=} \PY{l+s+s1}{\PYZsq{}}\PY{l+s+s1}{\PYZsq{}}
         
         \PY{k}{for} \PY{n}{key} \PY{o+ow}{in} \PY{n}{PoleisRawData2}\PY{p}{[}\PY{l+s+s1}{\PYZsq{}}\PY{l+s+s1}{Ionia}\PY{l+s+s1}{\PYZsq{}}\PY{p}{]}\PY{o}{.}\PY{n}{keys}\PY{p}{(}\PY{p}{)}\PY{p}{:}
             \PY{n}{ioniaText} \PY{o}{=} \PY{n}{ioniaText} \PY{o}{+} \PY{p}{(}\PY{n}{PoleisRawData2}\PY{p}{[}\PY{l+s+s1}{\PYZsq{}}\PY{l+s+s1}{Ionia}\PY{l+s+s1}{\PYZsq{}}\PY{p}{]}\PY{p}{[}\PY{n}{key}\PY{p}{]}\PY{p}{)}
\end{Verbatim}

    Train the tokenizer:

    \begin{Verbatim}[commandchars=\\\{\}]
{\color{incolor}In [{\color{incolor}53}]:} \PY{n}{trainedTokenizer} \PY{o}{=} \PY{n}{PunktSentenceTokenizer}\PY{p}{(}\PY{n}{ioniaText}\PY{p}{)}
\end{Verbatim}

    Apply it and print the results.

    \begin{Verbatim}[commandchars=\\\{\}]
{\color{incolor}In [{\color{incolor}54}]:} \PY{k}{for} \PY{n}{item} \PY{o+ow}{in} \PY{n}{trainedTokenizer}\PY{o}{.}\PY{n}{tokenize}\PY{p}{(}\PY{n}{hydisosText}\PY{p}{)}\PY{p}{:}
             \PY{n+nb}{print}\PY{p}{(}\PY{n}{item}\PY{p}{)}
             \PY{n+nb}{print}\PY{p}{(}\PY{l+s+s2}{\PYZdq{}}\PY{l+s+s2}{\PYZhy{}\PYZhy{}\PYZhy{}\PYZhy{}}\PY{l+s+s2}{\PYZdq{}}\PY{p}{)}
\end{Verbatim}

    \begin{Verbatim}[commandchars=\\\{\}]
Identifier: 891. , (Hydisseus) Map  61.  Lat. 37.10,long.
----
27.50. Size  of  territory:  ?
----
Type:  B:?
----
The  toponym  is  ` (Steph.
----
Byz. 645.17).
----
The  earliest  attestation  of  the  toponym is  in  a  C 1  inscription  (I.Stratonikeia 508.10  (c. 81)): `, although  it  was  mentioned  by  Apollonius  Aphrodisiensis, whose may  be  dated  to  C 3  (FGrHist 740  fr. 4).
----
The city-ethnic  is  `  (IG i³  265.ii.51;Apollonius Aphrodisiensis  (FGrHist 740)  fr. 4  (perhaps  C 3))  or `  (I.Mylasa 401.8  (C 2­C1)).
----
Hydisos  was  a  member  of  the  Delian  League,  but  is  registered  only  twice,  in  448/7  (IG i³  264.iii.21,  restored: `[\textasciitilde{}]) and 447/6  (IG i³  265.ii.51,  restored: `[\textasciitilde{}]),  paying  a  phoros  of 1  tal.
----
At  the  site  of  Hydisos  there  are  remains  of  city  walls  and towers,  probably  of  early  Hellenistic  date  (L.  Robert  ( 1935) 339­40).
----
890.
----
(Hymisseis) Map 61,  unlocated,  but  possibly  situat- ed  between  Amyzon  (no.
----
874)  and  Mylasa  (no.
----
913)  (Pontani ( 1997)  7;  cf.
----
L.  Robert  ( 1955)  226).
----
Type:  C: .
----
A  toponym  is not  attested.
----
The  city-ethnic  is  `µ  (IG i³  262.iv.19; restored:  [h ]µ\textasciitilde{}) or `µ  (IG i³  71.ii.143;  IG xii suppl.
----
127.58  (C 3)).
----
The  collective  use  of  the  city-ethnic  is attested  externally  in  the  tribute  lists  and  in  the  assessment decree  of 425/4  (IG i³  71.ii.143).
----
The  Hymisseis  were  members  of  the  Delian  League.
----
They are  recorded  three  times  in  the  lists  from  451/0  (IG i³ 262.iv.19)  to  447/6  (IG i³  265.ii.50),  paying  a  phoros  of 1,200 dr.
----
In  the  assessment  decree  of 425/4  they  form  a  syntely  with the  Edrieis  (no.
----
892)  and  Kyromeis,  and  the  three  together are  assessed  at  6  tal.  (IG i³  71.ii.143­44).
----
A  Hymesseus  is recorded  in a C 3  list  of  proxenoi  from  Eresos  (no.
----
796)  (IG xii suppl.
----
127.58).
----
891. 
----

    \end{Verbatim}

    For citations, we need to find matching brackets. This can be done using
SExprTokenizer.

    \begin{Verbatim}[commandchars=\\\{\}]
{\color{incolor}In [{\color{incolor}58}]:} \PY{k}{for} \PY{n}{i} \PY{o+ow}{in} \PY{n}{SExprTokenizer}\PY{p}{(}\PY{n}{strict}\PY{o}{=}\PY{k+kc}{False}\PY{p}{)}\PY{o}{.}\PY{n}{tokenize}\PY{p}{(}\PY{n}{hydisosText}\PY{p}{)}\PY{p}{:}
             \PY{k}{if} \PY{n}{i}\PY{p}{[}\PY{l+m+mi}{0}\PY{p}{]}\PY{o}{==}\PY{l+s+s1}{\PYZsq{}}\PY{l+s+s1}{(}\PY{l+s+s1}{\PYZsq{}}\PY{p}{:}
                 \PY{n+nb}{print}\PY{p}{(}\PY{n}{i}\PY{p}{)}
\end{Verbatim}

    \begin{Verbatim}[commandchars=\\\{\}]
(Hydisseus)
(Steph.  Byz. 645.17)
(I.Stratonikeia 508.10  (c. 81))
(FGrHist 740  fr. 4)
(IG i³  265.ii.51;Apollonius Aphrodisiensis  (FGrHist 740)  fr. 4  (perhaps  C 3))
(I.Mylasa 401.8  (C 2­C1))
(IG i³  264.iii.21,  restored: `[\textasciitilde{}])
(IG i³  265.ii.51,  restored: `[\textasciitilde{}])
(L.  Robert  ( 1935) 339­40)
(Hymisseis)
(no. 874)
(no. 913)
(Pontani ( 1997)  7;  cf.  L.  Robert  ( 1955)  226)
(IG i³  262.iv.19; restored:  [h ]µ\textasciitilde{})
(IG i³  71.ii.143;  IG xii suppl. 127.58  (C 3))
(IG i³  71.ii.143)
(IG i³ 262.iv.19)
(IG i³  265.ii.50)
(no. 892)
(IG i³  71.ii.143­44)
(no. 796)
(IG xii suppl. 127.58)

    \end{Verbatim}

    \subsection{Create dataframe}\label{create-dataframe}

    \begin{Verbatim}[commandchars=\\\{\}]
{\color{incolor}In [{\color{incolor}119}]:} \PY{n}{dfPoleisGesamt} \PY{o}{=} \PY{n}{pd}\PY{o}{.}\PY{n}{io}\PY{o}{.}\PY{n}{json}\PY{o}{.}\PY{n}{json\PYZus{}normalize}\PY{p}{(}\PY{n}{PoleisRawData2}\PY{p}{)}
\end{Verbatim}

    \begin{Verbatim}[commandchars=\\\{\}]
{\color{incolor}In [{\color{incolor}120}]:} \PY{n}{dfPoleisGesamt}\PY{o}{=} \PY{n}{dfPoleisGesamt}\PY{o}{.}\PY{n}{transpose}\PY{p}{(}\PY{p}{)}
          \PY{n}{dfPoleisGesamt}\PY{o}{.}\PY{n}{columns}\PY{o}{=}\PY{p}{[}\PY{l+s+s1}{\PYZsq{}}\PY{l+s+s1}{Beschreibung}\PY{l+s+s1}{\PYZsq{}}\PY{p}{]}
          \PY{n}{dfPoleisGesamt}\PY{o}{.}\PY{n}{head}\PY{p}{(}\PY{l+m+mi}{4}\PY{p}{)}
\end{Verbatim}

            \begin{Verbatim}[commandchars=\\\{\}]
{\color{outcolor}Out[{\color{outcolor}120}]:}                                                       Beschreibung
          Achaia.Ascheion  Identifier: 233. , (Ascheieus) Unlocated.  Typ{\ldots}
          Achaia.Boura     Identifier: 235. , (Bourios) Map  58.  Lat. 38{\ldots}
          Achaia.Helike    Identifier: 236. , (Helikeus) Map  58.  Lat. 3{\ldots}
          Achaia.Keryneia  Identifier: 237. , (Keryneus) Map  58.  Lat. 3{\ldots}
\end{Verbatim}
        
    \begin{Verbatim}[commandchars=\\\{\}]
{\color{incolor}In [{\color{incolor}121}]:} \PY{n}{dfPoleisGesamt}\PY{o}{=} \PY{n}{dfPoleisGesamt}\PY{o}{.}\PY{n}{reset\PYZus{}index}\PY{p}{(}\PY{p}{)}
          \PY{n}{dfPoleisGesamt}\PY{o}{.}\PY{n}{head}\PY{p}{(}\PY{p}{)}
\end{Verbatim}

            \begin{Verbatim}[commandchars=\\\{\}]
{\color{outcolor}Out[{\color{outcolor}121}]:}              index                                       Beschreibung
          0  Achaia.Ascheion  Identifier: 233. , (Ascheieus) Unlocated.  Typ{\ldots}
          1     Achaia.Boura  Identifier: 235. , (Bourios) Map  58.  Lat. 38{\ldots}
          2    Achaia.Helike  Identifier: 236. , (Helikeus) Map  58.  Lat. 3{\ldots}
          3  Achaia.Keryneia  Identifier: 237. , (Keryneus) Map  58.  Lat. 3{\ldots}
          4  Achaia.Leontion  Identifier: 238. , (Leontesios) Map  58.Lat.38{\ldots}
\end{Verbatim}
        
    \begin{Verbatim}[commandchars=\\\{\}]
{\color{incolor}In [{\color{incolor}122}]:} \PY{n}{dfPoleisGesamt}\PY{p}{[}\PY{l+s+s1}{\PYZsq{}}\PY{l+s+s1}{indexSplit}\PY{l+s+s1}{\PYZsq{}}\PY{p}{]} \PY{o}{=} \PY{n}{dfPoleisGesamt}\PY{p}{[}\PY{l+s+s1}{\PYZsq{}}\PY{l+s+s1}{index}\PY{l+s+s1}{\PYZsq{}}\PY{p}{]}\PY{o}{.}\PY{n}{str}\PY{o}{.}\PY{n}{split}\PY{p}{(}\PY{l+s+s1}{\PYZsq{}}\PY{l+s+s1}{.}\PY{l+s+s1}{\PYZsq{}}\PY{p}{)}
\end{Verbatim}

    \begin{Verbatim}[commandchars=\\\{\}]
{\color{incolor}In [{\color{incolor}123}]:} \PY{n}{dfPoleisGesamt}\PY{p}{[}\PY{l+s+s1}{\PYZsq{}}\PY{l+s+s1}{region}\PY{l+s+s1}{\PYZsq{}}\PY{p}{]} \PY{o}{=} \PY{n}{dfPoleisGesamt}\PY{p}{[}\PY{l+s+s1}{\PYZsq{}}\PY{l+s+s1}{indexSplit}\PY{l+s+s1}{\PYZsq{}}\PY{p}{]}\PY{o}{.}\PY{n}{apply}\PY{p}{(}\PY{k}{lambda} \PY{n}{raw}\PY{p}{:} \PY{n}{raw}\PY{p}{[}\PY{l+m+mi}{0}\PY{p}{]}\PY{p}{)}
          \PY{n}{dfPoleisGesamt}\PY{p}{[}\PY{l+s+s1}{\PYZsq{}}\PY{l+s+s1}{city}\PY{l+s+s1}{\PYZsq{}}\PY{p}{]} \PY{o}{=} \PY{n}{dfPoleisGesamt}\PY{p}{[}\PY{l+s+s1}{\PYZsq{}}\PY{l+s+s1}{indexSplit}\PY{l+s+s1}{\PYZsq{}}\PY{p}{]}\PY{o}{.}\PY{n}{apply}\PY{p}{(}\PY{k}{lambda} \PY{n}{raw}\PY{p}{:} \PY{n}{raw}\PY{p}{[}\PY{l+m+mi}{1}\PY{p}{]}\PY{p}{)}
          \PY{n}{dfPoleisGesamt}\PY{o}{.}\PY{n}{head}\PY{p}{(}\PY{p}{)}
\end{Verbatim}

            \begin{Verbatim}[commandchars=\\\{\}]
{\color{outcolor}Out[{\color{outcolor}123}]:}              index                                       Beschreibung  \textbackslash{}
          0  Achaia.Ascheion  Identifier: 233. , (Ascheieus) Unlocated.  Typ{\ldots}   
          1     Achaia.Boura  Identifier: 235. , (Bourios) Map  58.  Lat. 38{\ldots}   
          2    Achaia.Helike  Identifier: 236. , (Helikeus) Map  58.  Lat. 3{\ldots}   
          3  Achaia.Keryneia  Identifier: 237. , (Keryneus) Map  58.  Lat. 3{\ldots}   
          4  Achaia.Leontion  Identifier: 238. , (Leontesios) Map  58.Lat.38{\ldots}   
          
                     indexSplit  region      city  
          0  [Achaia, Ascheion]  Achaia  Ascheion  
          1     [Achaia, Boura]  Achaia     Boura  
          2    [Achaia, Helike]  Achaia    Helike  
          3  [Achaia, Keryneia]  Achaia  Keryneia  
          4  [Achaia, Leontion]  Achaia  Leontion  
\end{Verbatim}
        
    \begin{Verbatim}[commandchars=\\\{\}]
{\color{incolor}In [{\color{incolor}124}]:} \PY{n}{dfPoleisGesamt} \PY{o}{=} \PY{n}{dfPoleisGesamt}\PY{o}{.}\PY{n}{drop}\PY{p}{(}\PY{l+s+s1}{\PYZsq{}}\PY{l+s+s1}{index}\PY{l+s+s1}{\PYZsq{}}\PY{p}{,} \PY{l+m+mi}{1}\PY{p}{)}
          \PY{n}{dfPoleisGesamt} \PY{o}{=} \PY{n}{dfPoleisGesamt}\PY{o}{.}\PY{n}{drop}\PY{p}{(}\PY{l+s+s1}{\PYZsq{}}\PY{l+s+s1}{indexSplit}\PY{l+s+s1}{\PYZsq{}}\PY{p}{,} \PY{l+m+mi}{1}\PY{p}{)}
          \PY{n}{dfPoleisGesamt}\PY{o}{.}\PY{n}{head}\PY{p}{(}\PY{p}{)}
\end{Verbatim}

            \begin{Verbatim}[commandchars=\\\{\}]
{\color{outcolor}Out[{\color{outcolor}124}]:}                                         Beschreibung  region      city
          0  Identifier: 233. , (Ascheieus) Unlocated.  Typ{\ldots}  Achaia  Ascheion
          1  Identifier: 235. , (Bourios) Map  58.  Lat. 38{\ldots}  Achaia     Boura
          2  Identifier: 236. , (Helikeus) Map  58.  Lat. 3{\ldots}  Achaia    Helike
          3  Identifier: 237. , (Keryneus) Map  58.  Lat. 3{\ldots}  Achaia  Keryneia
          4  Identifier: 238. , (Leontesios) Map  58.Lat.38{\ldots}  Achaia  Leontion
\end{Verbatim}
        
    \subsubsection{Get city identifier}\label{get-city-identifier}

Throughout the full text, cities are referenced by a running index. To
make this information part of the dataframe, we extend it with an
additional column.

    \begin{Verbatim}[commandchars=\\\{\}]
{\color{incolor}In [{\color{incolor}125}]:} \PY{k}{def} \PY{n+nf}{cityIDFinder}\PY{p}{(}\PY{n}{text}\PY{p}{)}\PY{p}{:}
              \PY{l+s+sd}{\PYZsq{}\PYZsq{}\PYZsq{}}
          \PY{l+s+sd}{    \PYZsh{}1: Find all occurance of the string \PYZdq{}Identifier\PYZdq{} followed by a colon, a space and between one and four decimals.}
          \PY{l+s+sd}{    \PYZsh{}2: If there is a result, do the following}
          \PY{l+s+sd}{    \PYZsh{}3: Take the first result idList[0] (because the identifier is at the beginning of the text), }
          \PY{l+s+sd}{        and split the string at the dot (to remove the dot at the end of the string). Then return the string from the 13. position.}
          \PY{l+s+sd}{        This ensures, that only a number is returned, since it removes the word identifier, the colon, and the space. }
          \PY{l+s+sd}{    \PYZsq{}\PYZsq{}\PYZsq{}}
              \PY{n}{idList} \PY{o}{=} \PY{n}{re}\PY{o}{.}\PY{n}{findall}\PY{p}{(}\PY{l+s+s2}{\PYZdq{}}\PY{l+s+s2}{Identifier}\PY{l+s+s2}{\PYZbs{}}\PY{l+s+s2}{: }\PY{l+s+s2}{\PYZbs{}}\PY{l+s+s2}{d}\PY{l+s+s2}{\PYZob{}}\PY{l+s+s2}{1,4\PYZcb{}}\PY{l+s+s2}{\PYZbs{}}\PY{l+s+s2}{.}\PY{l+s+s2}{\PYZdq{}}\PY{p}{,} \PY{n}{text}\PY{p}{)} \PY{c+c1}{\PYZsh{}1}
              \PY{k}{if} \PY{n}{idList}\PY{p}{:} \PY{c+c1}{\PYZsh{}2}
                  \PY{n}{idCity} \PY{o}{=} \PY{n}{idList}\PY{p}{[}\PY{l+m+mi}{0}\PY{p}{]}\PY{o}{.}\PY{n}{split}\PY{p}{(}\PY{l+s+s1}{\PYZsq{}}\PY{l+s+s1}{.}\PY{l+s+s1}{\PYZsq{}}\PY{p}{)}\PY{p}{[}\PY{l+m+mi}{0}\PY{p}{]}\PY{p}{[}\PY{l+m+mi}{12}\PY{p}{:}\PY{p}{]} \PY{c+c1}{\PYZsh{}3}
                  \PY{k}{return} \PY{n}{idCity}
\end{Verbatim}

    \begin{Verbatim}[commandchars=\\\{\}]
{\color{incolor}In [{\color{incolor}126}]:} \PY{n}{dfPoleisGesamt}\PY{p}{[}\PY{l+s+s1}{\PYZsq{}}\PY{l+s+s1}{city\PYZus{}id}\PY{l+s+s1}{\PYZsq{}}\PY{p}{]} \PY{o}{=} \PY{n}{dfPoleisGesamt}\PY{p}{[}\PY{l+s+s1}{\PYZsq{}}\PY{l+s+s1}{Beschreibung}\PY{l+s+s1}{\PYZsq{}}\PY{p}{]}\PY{o}{.}\PY{n}{apply}\PY{p}{(}\PY{k}{lambda} \PY{n}{row}\PY{p}{:} \PY{n}{cityIDFinder}\PY{p}{(}\PY{n}{row}\PY{p}{)}\PY{p}{)}
\end{Verbatim}

    \subsubsection{Collection of all
citations}\label{collection-of-all-citations}

To collect all citations in the text for one city, we first use a
tokenizer from NLTK. This tokenizer collects all parenthesis and is much
easier to use, that regular expressions.

The basic assumption for citations is: They are written in parenthesis,
start with a capital letter, and contain at least one blank space (to
separate the authors name from text pages, indices, or dates).

    \begin{Verbatim}[commandchars=\\\{\}]
{\color{incolor}In [{\color{incolor}127}]:} \PY{k}{def} \PY{n+nf}{citationFinder}\PY{p}{(}\PY{n}{text}\PY{p}{)}\PY{p}{:}
              \PY{l+s+sd}{\PYZsq{}\PYZsq{}\PYZsq{}}
          \PY{l+s+sd}{    \PYZsh{}1: Generate a list of all capital letters}
          \PY{l+s+sd}{    \PYZsh{}2: Tokenize text to search for parenthesis, \PYZsq{}( ... )\PYZsq{} (this returns more accurate results, that using regular expression) }
          \PY{l+s+sd}{        The option strict=False is necessary to prevent errors, when the text contains only an unmatched opening or closing paranthesis.}
          \PY{l+s+sd}{    \PYZsh{}3: The basic assumptions for citations are: }
          \PY{l+s+sd}{        In parenthesis (first element is opening paranthesis), }
          \PY{l+s+sd}{        start with a capital letter (second element is a capital letter), }
          \PY{l+s+sd}{        and contain at least one blank space \PYZsq{} \PYZsq{}}
          \PY{l+s+sd}{    \PYZsq{}\PYZsq{}\PYZsq{}}
              \PY{k+kn}{import} \PY{n+nn}{string}
              \PY{n}{letters}\PY{o}{=}\PY{p}{[}\PY{n}{i} \PY{k}{for} \PY{n}{i} \PY{o+ow}{in} \PY{n}{string}\PY{o}{.}\PY{n}{ascii\PYZus{}uppercase}\PY{p}{]} \PY{c+c1}{\PYZsh{}1}
              \PY{n}{paranthesisTokenized} \PY{o}{=} \PY{n}{SExprTokenizer}\PY{p}{(}\PY{n}{strict}\PY{o}{=}\PY{k+kc}{False}\PY{p}{)}\PY{o}{.}\PY{n}{tokenize}\PY{p}{(}\PY{n}{text}\PY{p}{)} \PY{c+c1}{\PYZsh{}2}
              \PY{n}{listCite} \PY{o}{=} \PY{p}{[}\PY{n}{x} \PY{k}{for} \PY{n}{x} \PY{o+ow}{in} \PY{n}{paranthesisTokenized} \PY{k}{if} \PY{n}{x}\PY{p}{[}\PY{l+m+mi}{0}\PY{p}{]} \PY{o}{==} \PY{l+s+s1}{\PYZsq{}}\PY{l+s+s1}{(}\PY{l+s+s1}{\PYZsq{}} \PY{o+ow}{and} \PY{n}{x}\PY{p}{[}\PY{l+m+mi}{1}\PY{p}{]} \PY{o+ow}{in} \PY{n}{letters} \PY{o+ow}{and} \PY{l+s+s1}{\PYZsq{}}\PY{l+s+s1}{ }\PY{l+s+s1}{\PYZsq{}} \PY{o+ow}{in} \PY{n}{x}\PY{p}{]} \PY{c+c1}{\PYZsh{}3}
              \PY{k}{return} \PY{n}{listCite}
\end{Verbatim}

    \begin{Verbatim}[commandchars=\\\{\}]
{\color{incolor}In [{\color{incolor}128}]:} \PY{n}{dfPoleisGesamt}\PY{p}{[}\PY{l+s+s1}{\PYZsq{}}\PY{l+s+s1}{sources}\PY{l+s+s1}{\PYZsq{}}\PY{p}{]} \PY{o}{=} \PY{n}{dfPoleisGesamt}\PY{p}{[}\PY{l+s+s1}{\PYZsq{}}\PY{l+s+s1}{Beschreibung}\PY{l+s+s1}{\PYZsq{}}\PY{p}{]}\PY{o}{.}\PY{n}{apply}\PY{p}{(}\PY{k}{lambda} \PY{n}{row}\PY{p}{:} \PY{n}{citationFinder}\PY{p}{(}\PY{n}{row}\PY{p}{)}\PY{p}{)}
\end{Verbatim}

    \subsubsection{Transformation of
coordinates}\label{transformation-of-coordinates}

A simple regular expression is enough to find all coordinates in the
text. The coordinates are transformed from degrees/minutes to decimal to
enable plotting on a map with common projection.

    \begin{Verbatim}[commandchars=\\\{\}]
{\color{incolor}In [{\color{incolor}129}]:} \PY{k}{def} \PY{n+nf}{coordinateFinder}\PY{p}{(}\PY{n}{value}\PY{p}{,}\PY{n}{pattern}\PY{p}{)}\PY{p}{:}
              \PY{l+s+sd}{\PYZsq{}\PYZsq{}\PYZsq{}}
          \PY{l+s+sd}{    \PYZsh{}1: General function for finding regular expression pattern in a text.}
          \PY{l+s+sd}{    \PYZsh{}2: If patterns are found, do the following}
          \PY{l+s+sd}{    \PYZsh{}3: Take the last five values of the first string returned}
          \PY{l+s+sd}{    \PYZsh{}4: To convert angular in decimal coordinates: }
          \PY{l+s+sd}{        Take the returned value, split it at the dot}
          \PY{l+s+sd}{        convert the first part into a floating number (e.g. 36.0), }
          \PY{l+s+sd}{        and the second part into a integer number (e.g. 34) and divide it by 60. }
          \PY{l+s+sd}{        The sum the two results to return a coordinate in decimal system}
          \PY{l+s+sd}{    \PYZsq{}\PYZsq{}\PYZsq{}}
              \PY{n}{x} \PY{o}{=} \PY{n}{re}\PY{o}{.}\PY{n}{findall}\PY{p}{(}\PY{n}{pattern}\PY{p}{,} \PY{n}{value}\PY{p}{)}                                             \PY{c+c1}{\PYZsh{}1}
              \PY{k}{if} \PY{n}{x}\PY{p}{:}                                                                      \PY{c+c1}{\PYZsh{}2  }
                  \PY{n}{coord} \PY{o}{=} \PY{n}{x}\PY{p}{[}\PY{l+m+mi}{0}\PY{p}{]}\PY{p}{[}\PY{o}{\PYZhy{}}\PY{l+m+mi}{5}\PY{p}{:}\PY{p}{]}                                                      \PY{c+c1}{\PYZsh{}3 }
                  \PY{n}{decCord} \PY{o}{=} \PY{n+nb}{float}\PY{p}{(}\PY{n}{coord}\PY{o}{.}\PY{n}{split}\PY{p}{(}\PY{l+s+s1}{\PYZsq{}}\PY{l+s+s1}{.}\PY{l+s+s1}{\PYZsq{}}\PY{p}{)}\PY{p}{[}\PY{l+m+mi}{0}\PY{p}{]}\PY{p}{)} \PY{o}{+}  \PY{n+nb}{int}\PY{p}{(}\PY{n}{coord}\PY{o}{.}\PY{n}{split}\PY{p}{(}\PY{l+s+s1}{\PYZsq{}}\PY{l+s+s1}{.}\PY{l+s+s1}{\PYZsq{}}\PY{p}{)}\PY{p}{[}\PY{o}{\PYZhy{}}\PY{l+m+mi}{1}\PY{p}{]}\PY{p}{)}\PY{o}{/}\PY{l+m+mi}{60}   \PY{c+c1}{\PYZsh{}4}
                  \PY{k}{return} \PY{n}{decCord}
\end{Verbatim}

    \begin{Verbatim}[commandchars=\\\{\}]
{\color{incolor}In [{\color{incolor}130}]:} \PY{n}{dfPoleisGesamt}\PY{p}{[}\PY{l+s+s1}{\PYZsq{}}\PY{l+s+s1}{latitude}\PY{l+s+s1}{\PYZsq{}}\PY{p}{]} \PY{o}{=} \PY{n}{dfPoleisGesamt}\PY{p}{[}\PY{l+s+s2}{\PYZdq{}}\PY{l+s+s2}{Beschreibung}\PY{l+s+s2}{\PYZdq{}}\PY{p}{]}\PY{o}{.}\PY{n}{apply}\PY{p}{(}\PY{n}{coordinateFinder}\PY{p}{,} \PY{n}{pattern}\PY{o}{=}\PY{l+s+s2}{\PYZdq{}}\PY{l+s+s2}{Lat}\PY{l+s+s2}{\PYZbs{}}\PY{l+s+s2}{.}\PY{l+s+s2}{\PYZbs{}}\PY{l+s+s2}{s?}\PY{l+s+s2}{\PYZbs{}}\PY{l+s+s2}{d+}\PY{l+s+s2}{\PYZbs{}}\PY{l+s+s2}{.}\PY{l+s+s2}{\PYZbs{}}\PY{l+s+s2}{d+}\PY{l+s+s2}{\PYZdq{}}\PY{p}{)}
          \PY{n}{dfPoleisGesamt}\PY{p}{[}\PY{l+s+s1}{\PYZsq{}}\PY{l+s+s1}{longitude}\PY{l+s+s1}{\PYZsq{}}\PY{p}{]} \PY{o}{=} \PY{n}{dfPoleisGesamt}\PY{p}{[}\PY{l+s+s2}{\PYZdq{}}\PY{l+s+s2}{Beschreibung}\PY{l+s+s2}{\PYZdq{}}\PY{p}{]}\PY{o}{.}\PY{n}{apply}\PY{p}{(}\PY{n}{coordinateFinder}\PY{p}{,} \PY{n}{pattern}\PY{o}{=}\PY{l+s+s2}{\PYZdq{}}\PY{l+s+s2}{long}\PY{l+s+s2}{\PYZbs{}}\PY{l+s+s2}{.}\PY{l+s+s2}{\PYZbs{}}\PY{l+s+s2}{s*}\PY{l+s+s2}{\PYZbs{}}\PY{l+s+s2}{d+}\PY{l+s+s2}{\PYZbs{}}\PY{l+s+s2}{.}\PY{l+s+s2}{\PYZbs{}}\PY{l+s+s2}{d+}\PY{l+s+s2}{\PYZdq{}}\PY{p}{)}
\end{Verbatim}

    \subsubsection{Proper nouns}\label{proper-nouns}

To generate a list of all mentioned proper nouns for each city, we use
TextBlob. TextBlob is a NLTK tool with parts-of-speech tagger. We are
interessted in all parts that are `NNP' and longer then 3 letters.

This takes some time to process for the full dataframe. Behaviour can be
tested by uncommenting the cell below.

    \begin{Verbatim}[commandchars=\\\{\}]
{\color{incolor}In [{\color{incolor}131}]:} \PY{k}{def} \PY{n+nf}{namesFinder}\PY{p}{(}\PY{n}{text}\PY{p}{)}\PY{p}{:}
              \PY{l+s+sd}{\PYZsq{}\PYZsq{}\PYZsq{}}
          \PY{l+s+sd}{    \PYZsh{}1: Generate a blob out of the text }
          \PY{l+s+sd}{    \PYZsh{}2: Generate a list of all POS Tags, that are labeld as NNP(S) (Proper noun, singular (or plural)), and which are longer than 3 letters }
          \PY{l+s+sd}{    \PYZsq{}\PYZsq{}\PYZsq{}}
              \PY{n}{blobs} \PY{o}{=} \PY{n}{TextBlob}\PY{p}{(}\PY{n}{text}\PY{p}{)}                                                                              \PY{c+c1}{\PYZsh{}1}
              \PY{n}{namesList} \PY{o}{=} \PY{p}{[}\PY{n}{x}\PY{p}{[}\PY{l+m+mi}{0}\PY{p}{]} \PY{k}{for} \PY{n}{x} \PY{o+ow}{in} \PY{n}{blobs}\PY{o}{.}\PY{n}{pos\PYZus{}tags} \PY{k}{if} \PY{p}{(}\PY{n}{x}\PY{p}{[}\PY{l+m+mi}{1}\PY{p}{]} \PY{o}{==} \PY{l+s+s1}{\PYZsq{}}\PY{l+s+s1}{NNP}\PY{l+s+s1}{\PYZsq{}}\PY{p}{)} \PY{o}{|} \PY{p}{(}\PY{n}{x}\PY{p}{[}\PY{l+m+mi}{1}\PY{p}{]} \PY{o}{==} \PY{l+s+s1}{\PYZsq{}}\PY{l+s+s1}{NNPS}\PY{l+s+s1}{\PYZsq{}}\PY{p}{)} \PY{o+ow}{and} \PY{n+nb}{len}\PY{p}{(}\PY{n}{x}\PY{p}{[}\PY{l+m+mi}{0}\PY{p}{]}\PY{p}{)} \PY{o}{\PYZgt{}} \PY{l+m+mi}{3}\PY{p}{]}  \PY{c+c1}{\PYZsh{}2}
              \PY{k}{return} \PY{n}{namesList}
\end{Verbatim}

    \begin{Verbatim}[commandchars=\\\{\}]
{\color{incolor}In [{\color{incolor}134}]:} \PY{c+c1}{\PYZsh{} Uncomment to test routine. }
          
          \PY{n}{namesFinder}\PY{p}{(}\PY{n}{dfPoleisGesamt}\PY{p}{[}\PY{l+s+s1}{\PYZsq{}}\PY{l+s+s1}{Beschreibung}\PY{l+s+s1}{\PYZsq{}}\PY{p}{]}\PY{o}{.}\PY{n}{iloc}\PY{p}{[}\PY{l+m+mi}{10}\PY{p}{]}\PY{p}{)}
\end{Verbatim}

            \begin{Verbatim}[commandchars=\\\{\}]
{\color{outcolor}Out[{\color{outcolor}134}]:} ['Herodotos',
           'Stein',
           'Aigiroessa',
           'Elaia',
           'Herodotos',
           'Elaia',
           'Head',
           'Cook',
           'Aigiroessa',
           'Belkahve']
\end{Verbatim}
        
    \begin{Verbatim}[commandchars=\\\{\}]
{\color{incolor}In [{\color{incolor}135}]:} \PY{c+c1}{\PYZsh{}\PYZsh{}\PYZsh{}\PYZsh{}\PYZsh{}\PYZsh{}\PYZsh{}\PYZsh{}\PYZsh{}\PYZsh{}\PYZsh{}\PYZsh{}\PYZsh{}\PYZsh{}\PYZsh{}\PYZsh{}\PYZsh{}\PYZsh{}\PYZsh{}\PYZsh{}\PYZsh{}\PYZsh{}\PYZsh{}\PYZsh{}\PYZsh{}\PYZsh{}\PYZsh{}\PYZsh{}\PYZsh{}\PYZsh{}\PYZsh{}\PYZsh{}\PYZsh{}\PYZsh{}\PYZsh{}\PYZsh{}\PYZsh{}\PYZsh{}\PYZsh{}\PYZsh{}}
          \PY{c+c1}{\PYZsh{} Careful: Takes some time to evalute! \PYZsh{}}
          \PY{c+c1}{\PYZsh{}\PYZsh{}\PYZsh{}\PYZsh{}\PYZsh{}\PYZsh{}\PYZsh{}\PYZsh{}\PYZsh{}\PYZsh{}\PYZsh{}\PYZsh{}\PYZsh{}\PYZsh{}\PYZsh{}\PYZsh{}\PYZsh{}\PYZsh{}\PYZsh{}\PYZsh{}\PYZsh{}\PYZsh{}\PYZsh{}\PYZsh{}\PYZsh{}\PYZsh{}\PYZsh{}\PYZsh{}\PYZsh{}\PYZsh{}\PYZsh{}\PYZsh{}\PYZsh{}\PYZsh{}\PYZsh{}\PYZsh{}\PYZsh{}\PYZsh{}\PYZsh{}\PYZsh{}}
          
          \PY{n}{dfPoleisGesamt}\PY{p}{[}\PY{l+s+s1}{\PYZsq{}}\PY{l+s+s1}{names}\PY{l+s+s1}{\PYZsq{}}\PY{p}{]} \PY{o}{=} \PY{n}{dfPoleisGesamt}\PY{p}{[}\PY{l+s+s1}{\PYZsq{}}\PY{l+s+s1}{Beschreibung}\PY{l+s+s1}{\PYZsq{}}\PY{p}{]}\PY{o}{.}\PY{n}{apply}\PY{p}{(}\PY{k}{lambda} \PY{n}{row}\PY{p}{:} \PY{n}{namesFinder}\PY{p}{(}\PY{n}{row}\PY{p}{)}\PY{p}{)}
\end{Verbatim}

    \subsubsection{Cross links to other
cities}\label{cross-links-to-other-cities}

Links to other cities are mentioned in the fulltext with reference to
the index (e.g. `(no. 982)'). searching for these should give a link
list.

    \begin{Verbatim}[commandchars=\\\{\}]
{\color{incolor}In [{\color{incolor}136}]:} \PY{k}{def} \PY{n+nf}{linksFinder}\PY{p}{(}\PY{n}{text}\PY{p}{)}\PY{p}{:}
              \PY{l+s+sd}{\PYZsq{}\PYZsq{}\PYZsq{}}
          \PY{l+s+sd}{    \PYZsh{}1: Find all occurances of the string \PYZdq{}(no. 1234)\PYZdq{} with between one and four decimals}
          \PY{l+s+sd}{    \PYZsh{}2: If we have a result}
          \PY{l+s+sd}{    \PYZsh{}3: Generate a list, where every result:}
          \PY{l+s+sd}{        is split at the space, take the last part, and only up to the last letter (this removes the closing paranthesis) }
          \PY{l+s+sd}{    \PYZsh{}4: For all these results, convert the entries into an integer number}
          \PY{l+s+sd}{    \PYZsq{}\PYZsq{}\PYZsq{}}
              \PY{n}{x} \PY{o}{=} \PY{n}{re}\PY{o}{.}\PY{n}{findall}\PY{p}{(}\PY{l+s+s2}{\PYZdq{}}\PY{l+s+s2}{\PYZbs{}}\PY{l+s+s2}{(no}\PY{l+s+s2}{\PYZbs{}}\PY{l+s+s2}{. }\PY{l+s+s2}{\PYZbs{}}\PY{l+s+s2}{d}\PY{l+s+s2}{\PYZob{}}\PY{l+s+s2}{1,4\PYZcb{}}\PY{l+s+s2}{\PYZbs{}}\PY{l+s+s2}{)}\PY{l+s+s2}{\PYZdq{}}\PY{p}{,} \PY{n}{text}\PY{p}{)}               \PY{c+c1}{\PYZsh{}1}
              \PY{k}{if} \PY{n}{x}\PY{p}{:}                                                  \PY{c+c1}{\PYZsh{}2}
                  \PY{n}{links} \PY{o}{=} \PY{p}{[}\PY{p}{(}\PY{p}{(}\PY{n}{z}\PY{o}{.}\PY{n}{split}\PY{p}{(}\PY{l+s+s1}{\PYZsq{}}\PY{l+s+s1}{ }\PY{l+s+s1}{\PYZsq{}}\PY{p}{)}\PY{p}{)}\PY{p}{[}\PY{o}{\PYZhy{}}\PY{l+m+mi}{1}\PY{p}{]}\PY{p}{)}\PY{p}{[}\PY{p}{:}\PY{o}{\PYZhy{}}\PY{l+m+mi}{1}\PY{p}{]} \PY{k}{for} \PY{n}{z} \PY{o+ow}{in} \PY{n}{x}\PY{p}{]}     \PY{c+c1}{\PYZsh{}3}
                  \PY{n}{linksInt} \PY{o}{=} \PY{p}{[}\PY{n+nb}{int}\PY{p}{(}\PY{n}{x}\PY{p}{)} \PY{k}{for} \PY{n}{x} \PY{o+ow}{in} \PY{n}{links}\PY{p}{]}                 \PY{c+c1}{\PYZsh{}4}
                  \PY{k}{return} \PY{n}{linksInt}
\end{Verbatim}

    \begin{Verbatim}[commandchars=\\\{\}]
{\color{incolor}In [{\color{incolor}137}]:} \PY{n}{dfPoleisGesamt}\PY{p}{[}\PY{l+s+s1}{\PYZsq{}}\PY{l+s+s1}{linkedCities}\PY{l+s+s1}{\PYZsq{}}\PY{p}{]} \PY{o}{=} \PY{n}{dfPoleisGesamt}\PY{p}{[}\PY{l+s+s1}{\PYZsq{}}\PY{l+s+s1}{Beschreibung}\PY{l+s+s1}{\PYZsq{}}\PY{p}{]}\PY{o}{.}\PY{n}{apply}\PY{p}{(}\PY{k}{lambda} \PY{n}{row}\PY{p}{:} \PY{n}{linksFinder}\PY{p}{(}\PY{n}{row}\PY{p}{)}\PY{p}{)}
\end{Verbatim}

    \subsection{Display dataframe}\label{display-dataframe}

    \begin{Verbatim}[commandchars=\\\{\}]
{\color{incolor}In [{\color{incolor}138}]:} \PY{c+c1}{\PYZsh{} Uncomment to display full dataframe}
          \PY{c+c1}{\PYZsh{}df}
          \PY{n}{dfPoleisGesamt}\PY{o}{.}\PY{n}{head}\PY{p}{(}\PY{l+m+mi}{4}\PY{p}{)}
\end{Verbatim}

            \begin{Verbatim}[commandchars=\\\{\}]
{\color{outcolor}Out[{\color{outcolor}138}]:}                                         Beschreibung  region      city  \textbackslash{}
          0  Identifier: 233. , (Ascheieus) Unlocated.  Typ{\ldots}  Achaia  Ascheion   
          1  Identifier: 235. , (Bourios) Map  58.  Lat. 38{\ldots}  Achaia     Boura   
          2  Identifier: 236. , (Helikeus) Map  58.  Lat. 3{\ldots}  Achaia    Helike   
          3  Identifier: 237. , (Keryneus) Map  58.  Lat. 3{\ldots}  Achaia  Keryneia   
          
            city\_id                                            sources   latitude  \textbackslash{}
          0     233  [(CID ii  51.8  ( 339/8)), (BCH 45  ( 1921)  i{\ldots}        NaN   
          1     235  [(Morgan  and  Hall  ( 1996)  175;  Barr.), (R{\ldots}  38.166667   
          2     236  [(Morgan  and Hall ( 1996)  175;  Barr.), (Dio{\ldots}  38.250000   
          3     237  [(Rizakis  ( 1995)  206;  Barr.), (Paus. 7.25{\ldots}  38.166667   
          
             longitude                                              names  \textbackslash{}
          0        NaN  [Ascheieus, F.Delphes, Delphic, F.Delphes, Asc{\ldots}   
          1  22.250000  [Bourios, Keryneia, Paus, Strabo, Boura, Diako{\ldots}   
          2  22.166667  [Helikeus, Paus, Aigion, Strabo, Herakleides, {\ldots}   
          3  22.166667  [Keryneus, Paus, Keryneia, Mamousia, Derveni, {\ldots}   
          
                               linkedCities  
          0                            None  
          1  [236, 235, 238, 251, 165, 148]  
          2                       [231, 70]  
          3                           [353]  
\end{Verbatim}
        

    % Add a bibliography block to the postdoc
    
    
    
    \end{document}
